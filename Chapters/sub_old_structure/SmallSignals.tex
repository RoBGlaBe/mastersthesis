
\section{Small Signals}
\label{sec:SmallSignals}


* Small Signal Intro \\
* the smallest s2 signal that we can observe is a single electron being amplified \\
* the electron amp. gain is a detector-dependent parameter \\
* it states how many PE are detected for one SE dependent on the amp field \\
* a tilt or another irregularity of e.g. the gain- or anode-mesh results in a pos. dep. SEG \\
* measuring such a position dependence allows us to correct for it \\
* even though in this work this fails for two reasons we learn X (WHAT DID WE LEARN?)


\subsection{Theory}

* Sources of SEs \\
* related and unrelated SEs \\
* area of S2s $\propto$ amout of SEs \\
* Xenon plot (Alexey) \\
* Problem with Xebra: \\
** too little light for position reconstruction \\
** cant see SE at all bc self-trigger threshold too high \\
* pos reconstruction via NN: too little light


\subsection{Analysis}

* cuts: \\
* after s2 \\
* AFT \dots \\
* area-hist of ``SEs`` \\
* plot gaus from xenon it there too \\
* gains (ref to gains in intro) \\
* self-trigger treshold \\
* amplification process: path and chance to prod. photon \\
* almost never see single photons bc they are always under threshold \\
* percentage map \\
* conclusion: cannot see SEs


