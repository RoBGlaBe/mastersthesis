\chapter{Introduction}
\label{chap:Intro}

Intro Chap.-Intro

\FloatBarrier
\section{Dark Matter}
\label{sec:DarkMatter}
\FloatBarrier

Dark Matter:(see also other thesises) \\
* Evidence for DM \\
* direct detection


\newpage
\FloatBarrier
\section{Darwin}
\label{sec:Darwin}
\FloatBarrier

* Xenon evolution from 100 to nT to Darwin \\
* Why Darwin \\
* Why ultimate


\newpage
\FloatBarrier
\section{Research \& Development}
\label{sec:RnD}
\FloatBarrier

* What changes wrt XENONnT have to be made \\
* Challenges \\
* How do we look into that \\
* Lead up to xebra \\
* Need understanding: dual-phase \\
* Then we can compare single-phase and hermetic to that


\newpage
\FloatBarrier
\section{Xebra}
\label{sec:Xebra}
\FloatBarrier

Notes: \\
* Xebra (is not the TPC) \\
* Dual-Phase TPC \\
* Single-Phase \\
* Hermetic \\
* DAQ

The \gls{xebra} is the Freiburg based \darwin~R\&D platform for small scale \glspl{tpc}.
With \gls{xebra}, technical innovations are tested.
One being the hermetic \gls{tpc} where the \gls{lxe} at the outside of the sensitive volume of the \gls{tpc} is isolated from the sensitive part to reduce background interactions that originate inside the \gls{tpc}.
Another one is the single-phase \gls{tpc} where the \gls{gxe} for the secondary scintillation is replaced with \gls{lxe}.
This changes the secondary scintillation mechanism for which we need stronger electric fields that, among others, are achieved by thinner grid wires.
As a benchmark detector, our dual-phase \gls{tpc} relying on well understood detection and operation principles is employed first.
In this work, we investigate this benchmark \gls{tpc} and characterize it qualitatively.

%%%%% TPC
A light and charge sensitive \gls{tpc} requires a target material which is transparent to its own scintillation light and allows charge transport.
Typical scintillation materials are liquid Argon or \gls{lxe}.
Here, we use the latter.
A scintillation process via excimers as shown in fig.~\ref{fig:xe-scint-process} yields \gls{vuv} photons with a wavelength $ \lambda_\mathrm{scint} = \SI{178}{\nano\meter} $ as well as electron-ion pairs.  % TODO add plot!!
While the photons are detected instantly (S1 signal) the electrons drift towards the top of the \gls{tpc} due to the drift field applied.
The drift field is established by electrode grids - the cathode at the bottom of the \gls{tpc}, the anode at the top, and the gate \SI{5}{\milli\m} under the anode.
Once the electrons pass the gate, they enter the region of a stronger field.
In the middle between gate and anode, the phase changes from \gls{lxe} to \gls{gxe}.
The electrons arriving at the interface have a certain field-strength dependent probability to be extracted into the gas phase, where they eventually undergo proportional scintillation.  % TODO cite "Xu19"?
Every extracted electron is amplified via proportional scintillation to a larger number of photons, a certain fraction of which is detected by the \glspl{pmt}.
The number of \gls{pe} per extracted electron follows a Gaussian distribution with a mean of $ \mu \approx \SI{20}{\gls{pe}} $ and a standard deviation of $ \sigma \approx \SI{8}{\gls{pe}} $.
The timing information between S1 and S2 signal combined with the known drift velocity that depends on the drift field is further converted to vertical position $ z $ information giving the \emph{\glsentrydesc{tpc}} its name.

% XE scintillation excimers
\begin{figure}
\centering
\includegraphics[width=\figw\textwidth]{Figures/th.jpeg}  % {Figures/xe-scint.png}
\caption[Xenon Scintillation Process]{

    }
\label{fig:xe-scint-process}
\end{figure}


\paragraph{The light detection} is realized by the \glspl{pmt} that are mounted under the cathode (bottom \gls{pmt}) as well as above the anode (top \glspl{pmt}).
We are using one single $ 3\," $ Hamamatsu R114~10-21 \gls{pmt} as the bottom \gls{pmt} and seven $ 1\," $ Hamamatsu R85~20-406 \glspl{pmt}.
The positions of the top \glspl{pmt} are shown in fig.~\ref{fig:top-pmt-array}.  % TODO add plot
\gls{ptfe} reflectors form the radial boundary of the sensitive scintillation volume.
They are supposed to redirect as much of the isotropically emitted light as possible towards the \glspl{pmt}.
The mentioned parts are held in place by a support structure out of \gls{ptfe} and steel.
The choices of material and components are made to mimic optical attributes of larger detectors, where the choices are based on background minimization and low light loss.

% top pmt array
\begin{figure}
\centering
\includegraphics[width=\figw\textwidth]{Figures/th.jpeg}  % {Figures/top_pmt_array.png}
\caption[Top PMT array positions]{

    }
\label{fig:top-pmt-array}
\end{figure}


%%%%% DAQ
\paragraph{The \gls{daq}:} The read-out of the \glspl{pmt} is realized via an optional 10x~amplifier.
As sketched in fig.~\ref{fig:daq}, the analog \gls{pmt} output is read-out by a CAEN V1724 $ 8\;\mathrm{Channel} $ $ 14\,\mathrm{bit} $ \gls{adc} through an optional Phillips Scientific model 776 $ 16\;\mathrm{Channel} $ 10x~\gls{pmt} preamplifier.  % TODO add mca and amp specifications
Along with metadata and timing information, this data is then saved to disk for further software processing.


% daq sketch
\begin{figure}
\centering
\includegraphics[width=\figw\textwidth]{Figures/th.jpeg}  % {Figures/daq.png}
\caption[DAQ Sketch]{

    }
\label{fig:daq}
\end{figure}



%%%%% GAS SYSTEM
\paragraph{An external gas system} is used to fill the \gls{tpc} with Xenon.
Next to pumps, a key component of the gas system is the hot metal gas purifier or getter.
We pass Xenon through the getter in the filling process to clean it from impurities such as water and Oxygen.
While the experiment is running, Xenon is constantly taken out of the \gls{tpc} and conducted through the getter before it is lead back inside for continuous cleaning.
We call this process \emph{cycling}.
Every time the \gls{gxe} is cycled, it has to be cooled to become liquid.
As the cooling happens inside the \gls{tpc}, it has to be conducted without particle exchange.
Therefore, a liquid nitrogen cooled metal cavity filled from the top is energetically coupled to the top of the inside of the inner cryostat, called the coldfinger.  % mention cryostat above. explain outside structure in general!!
The inflow stream of \gls{gxe} is directed against the coldfinger cooling and liquefying the Xenon.
The complete gas system used is sketched in fig.~\ref{fig:gas-system}.


% gas system sketch
\begin{figure}
\centering
\includegraphics[width=\figw\textwidth]{Figures/th.jpeg}  % {Figures/gassystem.png}
\caption[Gas System Sketch]{

    }
\label{fig:gas-system}
\end{figure}


%%%%% SLOW CONTROL
\paragraph{The slow control} system Doberman \cite{} monitors critical parameters of the experiment and sends alarms to shifters once a parameter enters a predefined alarm region.
Examples of the monitored parameters are pressures in the \gls{tpc}, gas system and cryostat, temperatures at different points inside the \gls{tpc}, the heater's power consumption, and the levelmeters' read-outs inside the \gls{tpc} and the nitrogen storage.



\newpage
\FloatBarrier
\section{Strax}
\label{sec:Strax}
\FloatBarrier

* DAQ \\
* Straxbra / Strax \\
* Tuning of parameters \\
* Hermetic Context \\
* Tiny analysis?



The \gls{strax} encloses the essential software tools to process data stored on disk by the \gls{daq}.  % TODO add github cite
It is complemented by an experiment specific software package that defines the case specific usage of the tools provided by strax.
In the case of \oneton~and \nton~the package is straxen - \emph{streaming analysis for \textsc{Xenon}}.  % TODO add github cite
The \gls{xebra} equivalent to this, we call \emph{straxbra}.  % TODO add github cite

% straxbra data structure
\begin{figure}
\centering
\includegraphics[width=\figw\textwidth]{Figures/th.jpeg}  % {Figures/datastructure.png}
\caption[Data structure in straxbra]{

    }
\label{fig:data-structure}
\end{figure}


Starting at the lowest level \emph{raw records}, the data processing structure in straxbra follows a hierarchical order as shown in fig.~\ref{fig:data-structure}.
We call \emph{events}, \emph{peaks}, \emph{records}, and \emph{raw records} the \emph{data kinds} (colors in fig.~\ref{fig:data-structure}), which - in a processing sense - require one another in the mentioned order.
Data kinds describe different physical entities.
The records, e.g., are a list of data, each of them representing a \emph{record} which is a pulse of a single \gls{pmt} containing all the information there is about this record.
From records, we can produce peaks, which groups all records - or actually \emph{hits} which is an internal, hidden data kind\footnote{Anmerkung raus?} - that belong to one physical signal.
Due to the grouping, there is new information accessible such as the classification into S1 and S2 signal, the position of S2 signals, or the total light seen per peak.
Eventually, the peaks are grouped to form events that, in a physical sense, corresponds to an interaction.
When peaks are detected within about one maximal drift length, they can be correlated, i.e. belong to the same interaction.
An event usually contains at least one S1 and one S2 which lets us reconstruct e.g. the $ z $ position.

Data kinds are usually constituted of several, separately loadable parts e.g. \emph{peak~positions}, \emph{peak~classification}; see fig.~\ref{fig:data-structure}.
All these parts are built by so-called \emph{plug-ins} which are realized as python classes that inherit from the strax class \emph{plugin}.
The processing instructions of the data are specified in the plug-ins.
Every plug-in can provide several \emph{fields} holding information about every entry.
For example, the plug-in \emph{peaks} provides the following fields:
\begin{AutoMultiColItemize}
        \item{channel}
        \item{dt}
        \item{time}
        \item{length}
        \item{area}
        \item{area per channel}
        \item{n hits}
        \item{data}
        \item{width}
        \item{area decile from midpoint}
        \item{saturated channel}
        \item{n saturated channels}
\end{AutoMultiColItemize}
To allow traceable changes to plug-ins, every plug-in can be assigned any number of \emph{strax-options}.
Strax options have a default value that can optionally be run-dependent\footnote{muss ich das weiter erklären?} and is by default \emph{traced}.
This trace is an essential part of strax and is assigned to data that has been cached and stored.
The trace is a hash that is created out of the value of every strax-option the plug-in is assigned, the current plug-in's version, and the plug-ins it requires data from.
Like this, upon loading data of a plug-in, strax automatically reprocesses only the necessary data.


\paragraph{A \emph{context},} used to load data, holds all the specific information about what data and how the data is loaded.
The specifications include strax-options - when any other than the default value is used - other, constant parameters, the storage location, information about the database, and which plug-ins are loadable.
Thereby, for different \glspl{tpc} and different analyses, we can create dedicated contexts, which is more secure than changing all the options by hand every time for several reasons.
Naturally, the strax-options can and are supposed to change even within a context, since optimizing these parameters is an iterative process (parameter tuning).
Additional to the dual phase context, we thus created a context for the hermetic \gls{tpc} as well as one for the single-phase \gls{tpc}.

%%% explain data
In the following, we introduce the most used fields.

% data - waveform
\paragraph{The waveform} is labeled \emph{data} and shows the sampled \gls{pmt} output directly.
All the information is encoded and is extracted directly or indirectly from the waveforms.
The waveform of a peak can therefore be plotted as the number of \gls{pe} detected per sample against samples, an example is shown in fig.~\ref{fig:waveform}.
Since we fixed the number of samples saved to 220 and the \gls{adc} samples every \SI{10}{\nano\second}, we would be limited to a peak length of \SI{2.2}{\micro\second}.
S2 signals can extend a lot longer than that.
Therefore, the waveforms are downsampled to a sample size of an integer multiple of \SI{10}{\nano\second} until the peak fits in less than 221 samples.

\paragraph{The Area,} as visualized in fig.\ref{fig:area}, is the integral or rather the sum of the sample of a waveform and thus the area under the curve.
Hence, it is proportional to the amount of light seen by the \glspl{pmt} - or by one \gls{pmt} depending on what the data kind specifies.
The amount of light seen by a \gls{pmt} yields a proportional current, which is sampled with \SI{100}{\mega S\per\second} by the \gls{adc} into a $ 14\,\mathrm{bit} $ range.
The unit we measure the amount of light seen in has thus changed from a unit of current to \gls{adc}-units.
We use a single-photon light source to calibrate the \gls{pmt}-gains.
The gain of a \gls{pmt} determines the number of \gls{adc}-units measured with the respective \gls{pmt} for an average photon.
With the gains, we convert to the physically more meaningful unit \gls{pe}.
On average, \SI{1}{PE} corresponds to one detected photon.
In peaks and events, we use \gls{pe} units.
\emph{Area per channel} is integrating the \glspl{pmt} individually and yields one value per \gls{pmt}.

% width
\paragraph{The Width} of a peak is the extension of the central \SI{50}{\%} area quantile.
The center of the peak is the interpolated point where \SI{50}{\%} of the peak's area is reached.
The central $ x\,\% $ area quantile thus starts at the point where $ \nicefrac{x}{2}\,\% $ of the area is covered when integrating from the center to the left\footnote{technisch fehlt da ein vorzeichen. macht das was?}.
The quantile ends at the point where $ \nicefrac{x}{2}\,\% $ is covered integrating from the center to the right.
The time the quantile spans is the central $ x\,\% $ area quantile.

% aft
\paragraph{The \gls{aft}} specifies the fraction of area the top \gls{pmt} array contributes to a peak.
For example, \gls{aft} $ = 0 $ is a peak only seen in the bottom \gls{pmt}, whereas \gls{aft} $ = 1 $ is only seen in the top \gls{pmt}-array.
The \gls{aft} is strongly dependent on the incident position.
A Monte-Carlo simulation shows the trend in \gls{xebra} in fig.~\ref{fig:aft-mc}. % TODO cite alex
The \gls{aft} can be interpreted as the probability that a photon of an interaction at this point is seen in the top \gls{pmt}-array.
As S2 signals are produced in a thin slab they have a specific \gls{aft}\footnote{spezifizieren, nehme ich an... :(. soll ich den satz einfach weg lassen?}

% length (samples)
\paragraph{The length} is the number of samples a peak consists of.
A waveform always has \SI{220}{samples} for computational reasons.
Counting from the first (or programming counting zeroth), the length states how many of the \SI{220}{samples} do actually belong to the peak.

% dt - vllt zu length dazu... nee, doch nicht
\paragraph{dt or rather $ \mathbf{\Delta t}$} is the time resolution of the samples in a peak.
The \gls{adc} samples incoming currents with \SI{100}{\mega S\per\second}.
That translates to a sample every \SI{10}{\nano\second}.
As peaks are created out of records a waveform can be downsampled.
For computational reasons we want a peak to have a constant number of samples in a waveform, 220.
For larger peaks, we can give up time resolution and settle with a constant relative time resolution of a peak.
This does not change the general shape of the peak significantly.
Therefore, a peak is downsampled by an integer factor $ n $ so that the peak fits into 220 samples.
Effectively, this means summing up all $ n $ samples.
The time resolution hence is $ n \cdot \SI{10}{\nano\second} $.
Multiplying dt with the length in samples, we get the length of the peak in \SI{}{\nano\second}.

% risetime
\paragraph{The risetime} of a peak is the time from the \SI{10}{\%} to the \SI{50}{\%} area quantile.
The $ x\,\% $ area quantile being the point in the waveform at which $ x\,\% $ of the area is covered when integrating from the beginning of the peak.
As the name suggests, this field indicates the steepness of the peak's rising edge.
While S1s have rather shorter risetimes, S2s rise less steeply and thus have larger risetimes.

% n hits
\paragraph{N Hits} is the number of hits a peak consists of.
A hit is an occurrence above threshold and is a processing step between records and peaks.
In later versions of strax used in \nton, \emph{hits} is a data kind of its own.

% channel
\paragraph{A channel} in the context of \gls{xebra} simply means \gls{pmt}.
Any \emph{per channel} unit is also to be understood as \emph{per PMT}, e.g. \emph{area per channel} is the area measured in a specific \gls{pmt} and yields 8 values.

\paragraph{\emph{x} and \emph{y}} is the position in the horizontal plane.
A neural net trained on Monte Carlo data is used to determine the positions based on the light-share between the \glspl{pmt} in the top array.  % TODO cite alex thesis (key: ABism)
Only S2 peaks are used to determine the $ x $ and $ y $ positions as they are produced in a narrow $ z $ slab between liquid interface and anode, close to the top \glspl{pmt}.
Therefore, the light-share of S2s are more pronounced in \glspl{pmt} around the interaction point compared to the rather equal illumination in the case of S1s, as they are produced at the interaction point.


\paragraph{The \emph{z} position} can only be determined with the S1 and the corresponding S2 and therefore require the data kind \emph{event}.
We can infer the $ z $ position of the event from the time difference between the prompt S1 and the later S2 signal.
The detection time of S1 signals is the sum of photon propagation time and decay time or recombination time of the scintillating modes.
With $ n_\mathrm{\gls{lxe}} \approx 2 $, the photon propagation time $ t_\mathrm{S1} \approx \SI{0.36}{\nano\second} $ for one driftlength. % TODO cite "Solovov04" for lxe refr. idx = 1.56
The excimer decay has a slow ($ \tau_\mathr{S} = \SI{2.2}{\nano\second} $) and a fast component ($ \tau_\mathr{T} = \SI{34}{\nano\second} $).  % TODO cite "Kubota78" for decay time
Scintillation photons from electron-ion recombination are produced in the time scale of the recombination time $ \tau_\mathrm{R} = \SI{45}{\nano\second} $.  % TODO cite "Hitachi83" recomb. time
With the negligible propagation time, the detection time is in the order of a few \SI{10}{\nano\second} - independent of the interaction point.  % TODO cite again "Kubota78"
The S2's detection time is dominated by the drift time of the electrons that is with $ v_\mathrm{drift} \approx \SI{1.8}{\milli\meter\per\micro\second} $ in the order of a few \SIrange{1}{10}{\micro\second}.  % cite Jelles Thesis? "Aalbers18"
The detection time of S2s is thus by orders of magnitude larger than the S1's, while it is also dependant on the $ z $ position.
We compute $ z $ via $ z = - t_\mathrm{drift} v_\mathrm{drift} $, fixing $ t_\mathrm{drift} = 0 $ to $ z = 0 $ at the gate position.
Larger drift-times translate to positions deeper in the \gls{tpc} which correspond to negative $ z $.





Strax: ordering of S1 and S2 peaks in an event. Wieso hab ich das reingeschrieben? Vllt steht irgendwo etwas, was hier erklärt werden soll. Ist das schon durch obige erklärungen gecovert?
