\chapter{Introduction}
\label{chap:Intro}

Intro Chap.-Intro

\FloatBarrier
\section{Dark Matter}
\label{sec:DarkMatter}
\FloatBarrier

Dark Matter:(see also other thesises) \\
* Evidence for DM \\
* direct detection


\newpage
\FloatBarrier
\section{Darwin}
\label{sec:Darwin}
\FloatBarrier

* Xenon evolution from 100 to nT to Darwin \\
* Why Darwin \\
* Why ultimate


\newpage
\FloatBarrier
\section{Research \& Development}
\label{sec:RnD}
\FloatBarrier

* What changes wrt XENONnT have to be made \\
* Challanges \\
* How do we look into that \\
* Lead up to xebra \\
* Need understanding: dual phase \\
* Then we can compare singlephase and hermetic to that


\newpage
\FloatBarrier
\section{Xebra}
\label{sec:Xebra}
\FloatBarrier

Notes: \\
* Xebra (is not the TPC) \\
* Dual Phase TPC \\
* Single Phase \\
* Hermetic \\
* DAQ

\gls{xebra} is the Freiburg based \darwin R\&D platform for small scale \gls{tpc}s.
With it technical innovations are tested.
One being the hermetic \gls{tpc} where different \gls{lxe} volumes are separated to reduce background interactions that originate inside the \gls{tpc}.
Another one is the single phase \gls{tpc} where the gas phase is abandoned and the secondary scintillation of electrons takes place in the liquid phase which requires stronger fields.
As a benchmark a dual phase \gls{tpc} without inovations is employed first.
In this work we investigate this benchmark \gls{tpc}.

%%%%% TPC
The working principle of a \gls{tpc} is to have a scintillation material - in this case \gls{lxe}.
A scintillation process via excimers as shown in fig.~\ref{fig:xe-scint-process} yields UVU photons as well as electrons.  % TODO add plot!!
While the photons are detected instantly (S1 signal) the electrons drift towards the the top of the \gls{tpc} due to the drift field.
The drift field is established by electrode grids - the cathode at the bottom of the \gls{tpc}, the anode at the top and the gate \SI{5}{\milli\m} under the anode.
Once the electrons pass the gate, it enters the region of higher drift field.
In the middle between gate and anode the phase changes from \gls{lxe} to \gls{gxe}.
The electrons arriving at the interface have a certain probability to be extracted into the gaseous phase, where they eventually undergo proportional scintillation.
This effectively amplifies them to an average fix amount of photons per electron extracted (S2 signal).
The timing information between S1 and S2 signal combined with the known drift time that depends on the drift field is further converted to a vertical position ($ z $) information giving the \emph{time projection chamber} its name.


% XE scintillation excimers
\begin{figure}
\centering
\includegraphics[width=\figw\textwidth]{Figures/th.jpeg}  % {Figures/xe-scint.png}
\caption[Xenon Scintilltion Process]{

    }
\label{fig:xe-scint-process}
\end{figure}


\paragraph{The light detection} is realized by the \gls{pmt}s that are mounted under the cathode (bottom \gls{pmt}) as well as above the anode (top \gls{pmt}s).
We are using one single $ 3\," $ Hamamatsu R114~10-21 \gls{pmt} as the bottom \gls{pmt} and 7 $ 1\," $ Hamamatsu R85~20-406 \gls{pmt}s.
The positions of the top \gls{pmt}s is shown in fig.~\ref{fig:top-pmt-array}.  % TODO add plot
\gls{ptfe} reflectors mark the outside of the sensible scintillation volume in order to redirect as much light as possible towards the \gls{pmt}s.
The mentioned parts are hold in place by a support structure out of steel.
The choices of material are made such the intrinsic radiation and thus the background is reduced.

% top pmt array
\begin{figure}
\centering
\includegraphics[width=\figw\textwidth]{Figures/th.jpeg}  % {Figures/top_pmt_array.png}
\caption[Top PMT array positions]{

    }
\label{fig:top-pmt-array}
\end{figure}


%%%%% DAQ
\paragraph{The \gls{daq}:} The read-out of the \gls{pmt}s is realized via an optional 10x~amplifier
As sketched in fig.~\ref{fig:daq} the analogue \gls{pmt} output is read-out by a CAEN V1724 $ 8\;\mathrm{Channel} $ $ 14\,\mathrm{bit]} $ \gls{adc} through an optional Phillips Scientific model 776 $ 16\;\mathrm{Channel} $ 10x~pmt preamplifier.  % TODO add mca and amp specifications
Along with metadata and timing information this data is then saved to disk for further software processing.


% daq sketch
\begin{figure}
\centering
\includegraphics[width=\figw\textwidth]{Figures/th.jpeg}  % {Figures/daq.png}
\caption[DAQ Sketch]{

    }
\label{fig:daq}
\end{figure}



%%%%% GAS SYSTEM
\paragraph{An external gas system} is used to fill the \gls{tpc} with xenon.
Next to pumps and a key component of the gas system is the hot metal gas purifier or getter.
The getter is passed by the Xenon in the filling process in order to be cleaned from impurities such as water and Oxygen.
While the experiment is running, Xenon is constantly taken out of the \gls{tpc} and lead through the getter before it is lead back inside.
We call this process cycling.
Every time the \gls{gxe} has to be cooled in order to become liquid.
The cooling has to be done without particle exchange.
Therefore, a liquid nitrogen cooled metal cavity filled from the top is energetically coupled to the top of the inside of the inner cryostat called the coldfinger.  % mention cryostat above. explain outside structure in general!!
The inflow stream of \gls{gxe} is directed against the coldfinger cooling and liquefying the Xenon.
The complete gas system used is sketched in fig.~\ref{fig:gas-system}.


% gas system sketch
\begin{figure}
\centering
\includegraphics[width=\figw\textwidth]{Figures/th.jpeg}  % {Figures/gassystem.png}
\caption[Gas System Sketch]{

    }
\label{fig:gas-system}
\end{figure}


%%%%% SLOW CONTROL
\paragraph{The slow control} system Doberman \cite{} monitors critical parameters of the experiment and sends alarms once a parameter defines a predefined alarm region.
Some of the monitored parameters are pressures in the \gls{tpc}, gas system and cryostat, temperatures at different points inside the \gls{tpc} heater power consumption and levelmeters read-outs from inside the \gls{tpc} and in the nitrogen storage.



\newpage
\FloatBarrier
\section{Strax}
\label{sec:Strax}
\FloatBarrier

* DAQ \\
* Straxbra / Strax \\
* Tuning of parameters \\
* Hermetic Context \\
* Tiny analysis?



Strax: ordering of S1 and S2 peaks in an event
