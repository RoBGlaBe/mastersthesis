\FloatBarrier
\chapter{Analysis}
\label{chap:Analysis}
\FloatBarrier

Analysis Chap.-Intro

\FloatBarrier
\section{Single Electrons}
\label{sec:SE}
\FloatBarrier



\FloatBarrier
\subsection{Single Electron Production}
\label{ssec:se-source}
\FloatBarrier


A \gls{se} is one electron that is extracted from \gls{lxe} into the gas phase and is then amplified.
Light from S1s and S2s can photoionize impurities in \gls{lxe} that previously captured electrons.
If they are not captured by an impurity, they reach the liquid surface and undergo proportional scintillation in the case it is extracted into the gas phase.
The longer a \gls{se} drifts, the more likely it is recaptured.
Therefore, the \gls{se} rate drops with depth $ z $ or rather with time since last S2, as shown in fig.~\ref{fig:se-rate-xe100}.  % TODO add one of the plots of 1T, Xe100 or LUX from my SE talk
An S2 produces more light than the related S1 and has thus more potential to ionize.
The \gls{se} increases with larger S2 signals, i.e. higher energetic interactions and also with larger amounts of impurities\cite{}.  % TODO cite paper
Next to the ionization of impurities, there are other, secondary sources of \gls{se} like late extraction of electrons from the liquid surface.


% SE rate 1T, 100 or LUX
\begin{figure}
    \centering
    \includegraphics[width=0.95\textwidth]{Figures/se_rate_xe100.png}  % {Figures/th.jpeg}
    \caption[Single Electron Rate in 1T/100/LUX]{
        Single Electron rate in 1T/100/LUX. Currently Xenon100. Other? 2? All? Drawback with 1T: not a paper, but internal note but I think blessed...\cite{Aprile2014}.
    }
    \label{fig:se-rate}
\end{figure}


% TODO write which kind of run we use (krypton run 163)



\newpage

\FloatBarrier
\subsection{Single Electron Peak Selection}
\label{ssec:tagging}
\FloatBarrier


We discuss our peak selection cuts and \gls{se} selection here based on the example run 163.
We aim for a high sample purity with some compromise towards a large sample size.
The latter can be further improved by analyzing more data, e.g., longer runs.
Further reference to our \gls{se} population will be based on the selection discussed here.

\paragraph{Time after S2 Selection:} The longest drift of an electron occurs when it emerges just above the cathode.
Most \glspl{se} are thus expected within one drift length after large S2s, as photoionization by S2 light yields most \glspl{se}.
We therefore select all peaks in the window of one drift length after every large S2 peak in the run.
A large S2 in such a window closes the window early and the newly found S2 opens another window as usual.

\paragraph{Two Channels Cut:} By introducing a cut based on the field \emph{n channels}, we select peaks with a specific number of \glspl{pmt} contributing to it.
We demand at least two contributing \glspl{pmt} and thereby suppress dark counts as their coincidence rate is low.
Real \glspl{se} are large enough to be seen by at least two \glspl{pmt} most of the time and are therefore not affected by the cut.

\paragraph{\gls{aft} cut:} Although the meaning of \glspl{se} and S2s differ, \glspl{se} are produced via the same scintillation process as S2s and thus share some essential properties, like the specific \gls{aft}.
As S2 photons are produced in a narrow z-range close to the top \glspl{pmt}, they have a certain probability to be detected by the top rather than the bottom \gls{pmt}.
The result is a S2-specific \gls{aft}, $ \mathit{AFT}_\mathrm{S2} $.
We can see in fig.~\ref{fig:se-aft} that in the \gls{xebra} dual-phase \gls{tpc} $ \mathit{AFT}_\mathrm{S2} \approx 0.35 $.
However, the smaller the signal, the more the value is dominated by statistical fluctuations.
\glspl{se} in that sense behave like small S2s and have therefore a wide spread of \glspl{aft}.
Cutting on the contours given by $ a + \nicefrac{b}{\sqrt{\mathit{area}}} $ does not improve our \gls{se} population significantly, but would rather help with larger S2s.
We get better results with a flat \gls{aft} cut $ \mathit{AFT}_\mathrm{min} = 0.05 $ and $ \mathit{AFT}_\mathrm{max} = 0.95 $ cutting bigger population clearly not being \glspl{se}.

% SE AFT
\begin{figure}
    \centering
    \includegraphics[width=\figw\textwidth]{Figures/se_aft.png}  % {Figures/th.jpeg}
    \caption[AFT vs Area Single Electrons]{
        Single Electron AFT vs Area.
    }
    \label{fig:se-aft}
\end{figure}

% TODO weiter oben reinschreiben, welche areas wir erwarten & welches feld wir nutzen
The area histogram up to \SI{100}{\mathit{PE}} with all the previous cuts applied, fig.~\ref{fig:se-area-hist}, shows one broader population at just above \SI{20}{\mathit{PE}} with a long tail.
We identify this as our \gls{se} population where we find simultaneous occurrences of multiple electrons in the tail.
Another slightly overlapping population at areas smaller than \SI{10}{\mathit{PE}} is considered background, originating, e.g., from coincident dark counts.
Before fitting the population to extract the mean as the \emph{amplification gain} $ g_\mathrm{SE} $, we want to make sure that the background population does not influence the result of the fit.
We can check this by reducing the background using more stringent cuts with higher \gls{se} purity but lower tagging efficiency and compare the fit results.
A stable value hints towards a low impact of the population on the fit.

% SE area hist
\begin{figure}
    \centering
    \includegraphics[width=\figw\textwidth]{Figures/se_area_hist.png}  % {Figures/th.jpeg}
    \caption[Histogram Area Single Electrons and Background]{
        Single Electron AFT vs Area.
    }
    \label{fig:se-area-hist}
\end{figure}


We choose the reference regimes \emph{background}, \emph{\glspl{se}}, and \emph{tail} based on areas.
The representative regions are indicated in fig.~\ref{fig:se-area-hist} by the different background colors spanning from \numrange{0}{7} (background, green), \numrange{15}{30} (\gls{se}, red) and \numrange{40}{100}$ \,\mathit{PE} $ (tail, orange).
Furthermore, we compare other dimensions of the reference regimes to each other to find one where the backgrounds is well separable from the other two.
We peak length, risetime, and number of hits to show different behavior in the different reference regimes and thus make for possible cuts.
The respective histograms and cuts are shown in fig.~\ref{fig:len-cut-se},~\ref{fig:hit-cut-se} and \ref{fig:riset-cut-se}.


%  TODO add table as proposed in Fabians corrections





\newpage

\FloatBarrier
\subsection{Spatial Gain Dependence}
\label{ssec:g2}
\FloatBarrier

We fit all the resulting populations in an area histogram with % the model $ f_\mathrm{SE}\left( \ar \right) $

% se len cut
\begin{figure}
    \centering
    \includegraphics[width=\figw\textwidth]{Figures/se_cut_length.png}  % {Figures/th.jpeg}
    \caption[Length Histogramm and Cut for SEs]{
        Length Histogram of the previously defined reference regimes.
        The different reference regimes populate different length zones.
        This is used to introduce a cut to separate the background from the other two populations.
        The tail regime corresponds to coinciding few-electron signals and is thus later used to constrain the fit of the \gls{se} population.
        We therefore purposefully do not cut the tail but only define a minimal-length cut for the background indicated by the vertial blue line at \SI{350}{\nano\second}.
    }
    \label{fig:len-cut-se}
\end{figure}


% se n-hits cut
\begin{figure}
    \centering
    \includegraphics[width=\figw\textwidth]{Figures/se_cut_n_hits.png}  % {Figures/th.jpeg}
    \caption[N-Hits Histogramm and Cut for SEs]{
        N-Hits Histogram of the previously defined reference regimes.
        The different reference regimes populate different length zones.
        This is used to introduce a cut to separate the background from the other two populations.
        The tail regime corresponds to coinciding few-electron signals and is thus later used to constrain the fit of the \gls{se} population.
        We therefore purposefully do not cut the tail but only define a minimal-n-hits cut for the background indicated by the vertial blue line at $ 5\,\mathrm{hits} $.
    }
    \label{fig:hit-cut-se}
\end{figure}


% se risetime cut
\begin{figure}
    \centering
    \includegraphics[width=\figw\textwidth]{Figures/se_cut_risetime.png}  % {Figures/th.jpeg}
    \caption[Risetime Histogramm and Cut for SEs]{
        Risetime Histogram of the previously defined reference regimes.
        The different reference regimes populate different length zones.
        This is used to introduce a cut to separate the background from the other two populations.
        The tail regime corresponds to coinciding few-electron signals and is thus later used to constrain the fit of the \gls{se} population.
        We therefore purposefully do not cut the tail but only define a minimal-risetime cut for the background indicated by the vertial blue line at \SI{35}{\nano\second}.
    }
    \label{fig:riset-cut-se}
\end{figure}


% f_\mathrm{SE}\left( \ar \right) =
\begin{align}
    &\sum_{n} G \left(  \ar; n \cdot g_\mathrm{SE}, \sqrt{n} \cdot \sigma_\mathrm{SE}, A_n \right) +
    G_\mathrm{bg} \left( \ar;  \mu_\mathrm{bg}, \sigma_\mathrm{bg}, A_\mathrm{bg} \right), \\
    &\mathrm{where~} G \left( x;  \mu, \sigma, A \right) = A \cdot \exp{ \left\{ \frac{-\left( x - \mu \right)^2}{2\cdot\sigma^2} \right\} }
    \label{eq:se-fit-model}
\end{align}


is a Gaussian function, with $ n $ being the number of simultaneously detected electrons (here \numrange{1}{4}).
The parameter estimates of $ g_\mathrm{SE} $ by the fits in fig.~\ref{fig:se-area-fits} are hardly influenced by the different background cuts.
As the parameter is stable, we intend to reduce the background as much as possible for other analyses.
Our background cut hence is the n-hits cut, fig.~\ref{fig:hit-cut-se}.
The resulting fit parameter value is $ g_\mathrm{SE} = (22.11 \pm 0.05)\,\mathrm{PE} $.
The parameter is stable within $ 0.3\,\%_\mathrm{rel} $ in variation of the last background cut.


% area hist fit for g_se
\begin{figure}
    \centering
    \includegraphics[width=\figw\textwidth]{Figures/se-area-fits.png}  % {Figures/th.jpeg}  % TODO redo plot
    \caption[Area Histogram Fit of SEs]{
        Area histogram of the final \gls{se} population.
        The histogram includes the best fit of eq.~\ref{eq:se-fit-model} to the data between the two red lines indicating the fit window.
        A solid, orange line shows the best fit.
        The contributing Gaussian functions are shown separately as non-solid lines.
    }
    \label{fig:se-area-fits}
\end{figure}

With the same approach, we also investigate runs at different $ E_\mathrm{amp} $.
We compare the behavior to the one of \oneton~in fig.~\ref{fig:seg-vs-field}.
We note a systematically inferior light response of \gls{xebra}.  % TODO rephrase. we dont have a curve for xebra... just points
This can be explained by the different \gls{pmt} models of the top \gls{pmt} array, where the ones in \oneton~have the higher \glspl{qe} and \glspl{ce}.
The leveling of the liquid level can be conducted more precisely in a larger \gls{tpc} as well as better \gls{ptfe} reflectors are used in \oneton~ which increases the \gls{lce}.
% We measure $ g_\mathrm{SE, bot} = \,\mathrm{PE} $ at $ E_\mathrm{amp} \approx \SI{10.7}{\kilo\volt\per\centi\meter} $.
% We expect the signal seen in the bottom \gls{pmt} to be more comparable as we are using the better performing \gls{pmt} model and have a $ 100\,\% $ \gls{pmt} coverage at the bottom.


% g_se vs amp field
\begin{figure}
    \centering
    \includegraphics[width=\figw\textwidth]{Figures/se_gain_vs_field.png}  % {Figures/th.jpeg}
    \caption[\oneton~comparison of Amplification Gain vs. Fieldstrength]{
        $ g_\mathrm{SE} $ vs. $ E_\mathrm{amp} $
        % mention cite!!
    }
    \label{fig:seg-vs-field}
\end{figure}


Another issue we are facing lies in an earlier processing step of \emph{peaks}, the data kind used for this analysis.
In the waveform example in fig.~\ref{fig:waveforms-se}, we, on the one hand, note a frequent occurrence of samples with negative values and on the other hand a waveform that has seemingly arbitrarily split into two peaks.
The first issue can likely be traced back to the baseline reconstruction.
The second issue can likely be solved by refining the rules for peak splitting and peak merging.
Optimization of both these issues are possible, they are, however, out of the scope of this work.
While the first issue is responsible for an underestimate of nearly all peaks, the second can split up one \gls{se} with the full area into two which then share the total area of one physical \gls{se}.
Both effects reduce the value of $ g_\mathrm{SE} $ and the parameter is consequently underestimated.


% SE waveform problems
\begin{figure}
    \centering
    \includegraphics[width=\figw\textwidth]{Figures/th.jpeg}  % {Figures/se_waveforms.png}
    \caption[Single Electrons Waveforms]{
        Single Electron Waveforms
    }
    \label{fig:waveforms-se}
\end{figure}




It would be desirable to be able to resolve $ g_\mathrm{SE} $ spatially as a function of $ x $ and $ y $.
We would then be able to confirm a uniform amplification and to account for the irregularities.
As the spacial resolution is low for small signals~\cite{ABism} - the neural net is trained for $ > 100\,\mathrm{PE} $ - the position reconstruction is too imprecise in a small scale \gls{tpc} as \gls{xebra} to get a meaningful result.\footnote{auf alex thesis verweisen ohne weitere daten zu zeigen? haben wir nicht wirklich gecheckt, sondern uns auf darryls aussage verlassen}.




\newpage

\FloatBarrier
\subsection{Intrinsic Width of Single Electrons}
\label{ssec:width}
\FloatBarrier


As discussed in sec.~\ref{ssec:se-source}, the width is very sensitive to the size of the gas gap, and therefore the width is a good indicator of deformations of the grid~\cite{?}.
A spacial mapping is not possible due to the lack of a position reconstruction for \glspl{se}.

% S2s usually have $ z $ dependent widths due to longitudinal diffusion.
% The arrival time at the liquid interface of the electrons in one S2 electron clound spreads dependent on the drift time.
Due to longitudinal diffusion, S2 widths depend, among others, on the interaction depht $ z $, or the drift time.
The spread of the electron cloud in \gls{lxe} increases accordingly with $ z $ when arriving at the liquid interface.
For \glspl{se} this is not the case since these are single quanta.
Thus, \glspl{se} tell us the intrinsic width of one electron causing scintillation which can be used to infer the exact time between the first and last electron in the electron cloud.




\newpage



\subsection{rate}
\label{ssec:rate}



\newpage

\FloatBarrier
\section{Krypton}
\label{sec:Kr}
\FloatBarrier

With an appropriate source we can correct position dependent signal loss and further convert to an absolute energy scale.
The source of choice is an excited Krypton state.
We need to prepare the recorded signal by tagging the data from the source in order to investigate signal losses due to \gls{lce} and \gls{elt}.
With the correction of these losses we are able to convert to the absolute energy scale.
% depending of what else I write in the chapter, add more description here



\FloatBarrier
\subsection{source}
\label{ssec:source}
\FloatBarrier


describe how we can fill the tpc with krypton here.. whole cycle: rubidium, gas system, etc..


Rubidium \isotope[83][37]{Rb} decays via two metastable excitations to the stable \isotope[83][36]{Kr} Krypton isotope.
It first transitions via electron capture to \isotope[83m2]{Kr}.
This state has a half-life of \SI{1.83}{\hour} and decays with an energy emission of \SI{32.1}{\kilo\electronvolt}, see fig.~\ref{fig:scheme_kr}.
Under the emission of an electron from internal conversion and subsequent emission of auger electrons and small amounts of X-rays its core is transfered to another exited state \isotope[83m1]{Kr}.
The de-excitation to the stable \isotope[83]{Kr} is mostly via internal conversion and in small amounts via gamma-rays.
The half-life here is \SI{154}{\nano\s} and the decay energy is \SI{9.4}{\kilo\electronvolt}.


% Krypton Decay Scheme
\begin{figure}
    \centering
    \includegraphics[width=0.95\textwidth]{Figures/th.jpeg}  % {Figures/scheme_kr_decay.png}
    \caption[Decay Scheme Krypton]{Kr-decay scheme\cite{kr_scheme}.}
    \label{fig:scheme_kr}
\end{figure}

% want kr inside tpc
The solid Rubidium source is placed in a reservoir inside the gas system.
The flow can be lead through the reservoir or parallel to it without contact to it.
When the flow follows the line that goes through the reservoir the gaseous \isotope[83m2]{Kr} that stems from the Rubidium decay is carried along with the Xenon into the \gls{tpc}.
There, it disperses and is spread homogeneously within approximately a minute.
Once it decays, the second decay takes place virtually immediately respectively resulting in \gls{lxe} scintillation, sec.~\ref{sec:Xebra}.
Fig.~\ref{fig:waveform_kr} shows a sample waveform of this process.
The different channels of both decays - even though the energy is the same - result in slightly different light and electron responses in the scintillation process. (TODO: scint response too slangy?)
This is shown in fig.~\ref{fig:scint_response_nest}.
We are not able to resolve these subtle differences with other statistical effects limiting us.
It will, however, contribute to the broadening of the energy resolution. (TODO: energy resolution too slangy?)
As the final decay product is \isotope[83]{Kr}, a stable Noble gas, we do not get unwanted effects when leaving it in the Xenon.
A very conservatively estimated \SI{50000}{} atoms per filling and the Avogadro constant $ \mathrm{N}_\mathrm{A} \approx 6.02\cdot10^{23}\,\mathrm{mol}^{-1} $ yields a sub femto mol level after 1000 fillings.
At this level the macroscopic properties of Xenon are not altered by the remaining Krypton.


% Kr sample waveform
\begin{figure}
    \centering
    \includegraphics[width=0.95\textwidth]{Figures/th.jpeg}  % {Figures/waveform_kr.png} TODO make this plot and save it to figs-dir
    \caption[Waveform Krypton]{sample waveform of krm2 to krm1 and krm1 to kr. s1, s2... etc.}
    \label{fig:waveform_kr}
\end{figure}


% Nest scint response
\begin{figure}
    \centering
    \includegraphics[width=0.95\textwidth]{Figures/th.jpeg}  % {Figures/nest_scint_response.png}
    \caption[Nest scint response]{Nest scint response\cite{}.} % TODO cite nest paper and write subbing
    \label{fig:scint_response_nest}
\end{figure}


With the \SI{50000}{} initial radioactive nuclei of Krypton, after 7 half-lives less than 400 nuclei that have not decayed yet are left.
This corresponds to a sub-percent level after less than \SI{13}{\hour}. \footnote{TODO: Sounds weird to mention half-life in the same sentence as half-lifes and hours in the same sentences as per-cent-level, but I don't want to sound stupid by saying sub-percent corresponding to 400 nuclei. this is the true value after 7 half-lives, but doesnt correspond to the percent level, which seems to be what I'm implying then. on the other hand I want to make clear, that we always have a sub-percent level after 7 half-lives, that doesnt depend on the half-life. ideas? or maybe just leave like that?}
After a day we can thus easily take data without a noticeable amount of Krypton decays occurring.

% TODO the following has been written before the text above. it might need to be worked on again in order to not repeat to much.
% also the desired rate is calculated differently. needs to be changed.
\paragraph{Krypton runs} are what we henceforth call runs we filled \gls{tpc} with the metastable Krypton Isotope \isotope[83m2][36]{Kr} for.
We aim to record a maximum number of these Krypton decays in the \SI{3}{\min} recordings.
The signals of two or more interactions at the same time - meaning an interaction happened before or just at the same time as the previous \st~ is registered - cannot unambiguously assigned to its correct physical counterpart.
Therefore, we can approximate an optimal rate of interactions and accordingly activity of our Krypton source $ A_\mathrm{Kr} $ inside the \gls{tpc}.
To do that we demand the probability of exactly one decay within a given time window to be maximal.
With the Krypton being distributed homogeneously within the \gls{tpc} the average \st~ drift time is half the maximum drift time of about $ \SI{20}{\micro\s}$.
Thus the optimal activity of Krypton can be calculated using Poissonian statistics with $\lambda = \SI{20}{\micro\s} \cdot A_\mathrm{Kr}$ and $ k = 1 $:

\begin{equation}
    P_{\lambda}\left(k\right) =  \frac{\lambda^k}{\exp^{-\lambda}}
\end{equation}

The highest probability to have exactly one decay within the average drift time of \SI{20}{\micro\s} is exactly \SI{50}{\kilo\becquerel}.
Although \SI{40}{\micro\s} is a good estimate, our maximum drift time depends on the drift field and can thus vary from run to run, depending on the field we choose.
It is also not easy to adjust how much Krypton exactly is let into the \gls{tpc} and thus the activity inside.
This is not necessary, because of the short decay time the activity will decrease quickly.
At the same time the probability around the maximum is rather flat so the amount of events that can be unambiguously reconstructed does not change drastically.
On the on hand, lesser activity will lead to less decays happening at the same time, but at the same time to a lot of time in which no decay will occur.
On the other hand, a greater activity increases overlapping events.
This calculation assumes having overlapping events is as unappreciated as having no data in a certain time slot.
Overlapping events can lead to mistakes in the reconstruction and are thus rather to be avoided than the other case.
The calculation also assumes only Krypton interactions to occur, whereas we know we also have other signal happening simultaneously.
For these reasons we choose a activities smaller than the theoretical optimum.


One minute after starting the filling process of Krypton for a Krypton run it reaches the inner part of the \gls{tpc}.
We allow the Krypton to disperse in the liquid Xenon for five minutes to ensure that is distributed homogeneously inside the \gls{tpc}.
That this is enough time to ensure a homogeneous distribution will be shown in subsection~\ref{ssec:tagging}.



\FloatBarrier

\newpage

\FloatBarrier
\subsection{tagging}
\label{ssec:tagging}
\FloatBarrier



The calibration of the XeBRA \gls{tpc} requires data from homogeneously distributed Krypton as we have in our Krypton runs.
We have to clean the data from these runs from every data that does not originate in the decay of the metastable Krypton we filled in.
Apart from Krypton decay data, we will find data from other background sources, dark counts, noise and others that we will discuss, in Krypton runs.
The aim is to reduce any non-signal contributions to a minimum while keeping as much signal as possible and without biasing the that data by tagging a non-representative subset of the signal.
Although we will show that we can only tag a subset of our signal with a decay time of \SI{300}{\nano\s} and greater, this subset is representative in any other concern.
We will show that we can achieve a rejection of \dots and acceptance of \dots.  % TODO fill in rejection and acceptance
To illustrate the selection of data of this kind we choose run 250 as an example.
The cuts we discuss here are applied in the same way to any other Krypton run.


The mean time difference between the two consecutive Krypton decays is the lifetime $\tau_\mathrm{b} = \SI{222.8}{\nano\s}$.
Therefore, both decay signals are reconstructed into the same event.
Due to the ordering of peaks explained in sec.~\ref{sec:Strax} the \SI{32}{\kilo\eV} is the \gls{s1a}.
Following this convention we will name the \gls{s1b}, the \gls{s2a} and the \gls{s2b}, respectively.
We start our event selection by using the striking feature of the two consecutive decays in one event.


\paragraph{The two decays cut} firstly demands two S1 peaks to be present in the event.
We cannot demand the same for S2 peaks, since \gls{s2a} and \gls{s2b} overlap in the vast majority of events, because of the greater intrinsic width of S2 signals.
We thus see the two S2 signals as one combined S2 peak and cannot extract information about just one of them.
This is why we can secondly only demand one S2 instead of two.
Therefore, our cut could be $n_\mathrm{peaks} \ge 3$ (sec.~\ref{sec:Strax}).
There are, however, two drawbacks compared to the cut we use instead.
Fist, we are not guaranteed two S1s and one S2, but just three peaks in total.
Theoretically, three S2 signals would satisfy the condition.
And second, a peak can be counted into $ n_\mathrm{peaks} $, even though its waveform integrates to a negative or zero area.
If on the other hand, no \gls{s1b} is present in an event its area will be set to zero.
Both drawbacks can be avoided by exploiting that non-present peaks have zero area.
Our cut condition is that the areas of all, \gls{s1a}, \gls{s1b} and \gls{s2a} are positive.

This cuts rejects \SI{58.1}{\%} of \emph{all} events. % while the acceptance should be \SI{100}{\%} at first glance.
% However, as we will show, due to the current peak finder in strax, \gls{s1a} and \gls{s1b} are separable if they are at least \SI{300}{\nano\s} apart.
% Events with a distance lower than that are falsely rejected.
% We thus have an optimal overall acceptance of an estimated \SI{20}{\%}.
% The current cut does not account for all of the falsely rejected events.
The weaker statement cuts only \SI{14.9}{\%} while it does not remove a single event the stronger condition would not remove, too.


\paragraph{The time ordering cut} removes events in case they do not reflect the correct ordering of the decay chain.
We base this cut on the time difference of \gls{s1a} and \gls{s1b}.
Since the index ordering is based on the area of the respective peak, we have to make sure the larger peak is detected first.
For this we define the time difference between them as $t_\mathrm{diff} = t_{\mathrm{S}1_\mathrm{b}} - t_{\mathrm{S}1_\mathrm{a}}$.
We demand that $t_\mathrm{diff}$ is positive.
It is worth noting that $t_\mathrm{diff}$ is the decay time $\tau_\mathrm{b}$ of \isotope[83m1]{Kr} that we will use again later. % ('later' ok? sounds like temp ordering)

We purposely do not demand a proportion of $ \nicefrac{32}{9} $ - the proportion of their decay energies - between the areas of the two S1s.
This condition requires stronger assumptions.
Therefore, we do not want to enforce this condition here, but remove such misreconstruction later otherwise.
The energy ration hypothesis remains to be tested.

% Z-Cut
\begin{figure}[H]
    \centering
    \includegraphics[width=0.95\textwidth]{Figures/th.jpeg}  % {Figures/z-cut.png}
    \caption[Fiducial z-cut]{Histogram of the event z positions of all events of an example Krypton run.
    All events missing either S1s or S2s effectively reconstruct to a z-position of zero explaining the high overshoot at this position.
    The second bump is at the gate position of $ z = \SI{-2.5}{mm}$ and background events originating there.
    These events are suppressed by a fiducial cut at $ z = \SI{-8}{mm}$ indicated by the red dotted line on the right.
    The dotted red line on the left side indicates the lower $z$ cut at $ z = \SI{-68}{mm}$ preventing similar background contributions from the cathode grid,
    even though they are not as dominantly visible.
    The sudden drop-off at below $ z = \SI{-70}{mm}$ indicates the maximum drift length of the \gls{tpc}.
    }
    \label{fig:fid-z-cut}
\end{figure}

\paragraph{The fiducial volume cut} rejects all events that have been reconstructed into a specified volume in the center of the drift volume.
\emph{Fiducialization} is the self shielding property of liquid Xenon.
In large scale \gls{tpc}s like \nton ~this cut is supposed to cut background $\alpha$ and $\beta$ decays from radioactive contamination of the electrodes or the PTFE reflectors, further referred to as wall events.
In these experiments with science goals these volumes are selected very carefully to minimize the background while achieving maximum exposure to reach higher statistics or alternatively stronger exclusion limits.
These fiducial volumes are typically described by higher order polynomials on the basis of Monte-Carlo simulations. %TODO cite paper that shows fid vol cut with polynomials (xenon1t?)
This kind of effort cannot be justified for the goals of this analysis.
Therefore, the description is not optimized for high exposure, but only for background exclusion.
At the same time, this justifies an easier modeling of the fiducial volume.
We define a cylindrical volume by a maximal and minimal value for the height $z$ and a maximal radius $r_\mathrm{max} = \SI{23}{\milli\m}$.
While the $z$ cut of $z_\mathrm{min}= \SI{-68}{mm}$ and $z_\mathrm{max}= \SI{-8}{mm}$ is chosen based on data as shown in fig.~\ref{fig:fid-z-cut} and has only the goal of just mentioned background exclusion, we have further restrictions for the radius.
The position reconstructed by the neural net is only accurate within this radial limit of $r_\mathrm{max}$\cite{ABism}.


% Area-Width after fid.
\begin{figure}[H]
\centering
\includegraphics[width=0.95\textwidth]{Figures/th.jpeg}  % {Figures/oterhS1_area_width_after_fid.png}
\caption[Area-Width Histogram smaller S1 after Fid. Cut]{
        Sample Text.
        Sample Text..
        Sample Text...
    }
\label{fig:other_s1_area_width}
\end{figure}


\paragraph{The false smaller S1 pairing cut} removes events with apparent misclassification of \gls{s1b} peaks.
Besides the existence and time ordering we did not define any restrictions based on the \gls{s1b} peak.
This allows pairing of coincident S1s of any kind, as long as they have the correct time ordering.
In fig.~\ref{fig:other_s1_area_width} we see the second largest S1 area-width histogram of events that satisfy all previous selection criteria.
Although, we expect a single population due to the mono energetic decay of Krypton we see two separate populations in the plot.
One population has both lower width and lower areas than the other.
The width (sec.~\ref{sec:Strax}) of about \SI{10}{\nano\s} corresponds to waveforms which have most of their area in one sample which is not what we expect from real S1s.
Also, the area of \numrange{3}{20}$\,\mathrm{PE}$ is less than what we expect from simulations.
Both statements disqualify this population which we thus exclude with the following cuts based on figs.~\ref{fig:other_s1_area_cut},~\ref{fig:other_s1_width_cut}.
With the slightly conservative cuts at $30\,\mathrm{PE}$ area and \SI{24}{\nano\s} width of the second largest S1 we remove most mismatches.
To further clean up the Krypton selection we investigate other dimensions (better word?).


% Area Hist Cut
\begin{figure}[h]
\centering
\includegraphics[width=0.95\textwidth]{Figures/th.jpeg}  % {Figures/oterhS1_area_hist_cut.png}
\caption[Other S1 Area Histogram Cut]{
        Sample Text.
        Sample Text..
        Sample Text...
    }
\label{fig:other_s1_area_cut}
\end{figure}


% Width Hist Cut
\begin{figure}
\centering
\includegraphics[width=0.95\textwidth]{Figures/th.jpeg}  % {Figures/oterhS1_width_hist_cut.png}
\caption[Other S1 Width Histogram Cut]{
        Sample Text.
        Sample Text..
        Sample Text...
    }
\label{fig:other_s1_width_cut}
\end{figure}


\paragraph{The false main S1 paring cut}, similarly to the \emph{false smaller S1 pairing cut} restricts the \gls{s1a} peak to avoid misclassification.
In fig.~\ref{fig:main_s1_area_width} there are to distinct populations.
The one to the left spans about \SI{400}{PE} and is broader width distribution.
While the one on the right spans about \SI{9000}{PE} at higher areas and with a narrower widths distribution.
The decay of \isotope[83m2]{Kr} that we want to tag as \gls{s1a} is mono energetic and does thus not comply with the larger spread of \SI{9000}{PE}.
We remove this population with a cut in area such that we will not bias real Krypton S1 population in other runs we use this cut with.
In accordance with fig.~\ref{fig:main_s1_area_width}, we place the cut at $ area_\mathrm{max} = \SI{600}{PE}$.

We assume that our selection is completed.
This is tested against the homogeneity premise and against the decay time of \isotope[83m1]{Kr}, $\tau_\mathrm{b} = \SI{222.8}{\nano\s}$.
The remaining events are referred to as Krypton events.


% Area-Width main S1
% This plot is missing!!!
% TODO: plot aus notebook speichern und hier einfügen...
\begin{figure}
\centering
\includegraphics[width=0.95\textwidth]{Figures/th.jpeg}  % {Figures/S1_area_width_after_fid.png}
\caption[Area-Width Histogram of main S1 after Fid. Cut]{
        Sample Text.
        Sample Text..
        Sample Text...
    }
\label{fig:main_s1_area_width}
\end{figure}


\paragraph{A homogeneous spacial distribution} of Krypton inside the \gls{tpc} is to be expected.
With the tagging of Krypton via the discussed cuts in place we can test this hypothesis against the data to test our selection.
In the histogram of events over equal volume spaced r-bins, fig.~(TODO ref. r plot here) we observe counts inside the fiducial r-cut of $r_\mathrm{max} = \SI{23}{\milli\m}$ that is agreement with homogeneity.
The bins with the smallest r-values show higher counts and would thus not confirm homogeneity.
However, we know that the neural net responsible for the position reconstruction is has a tendency of projecting the position inwards. % TODO cite alex bism
It is thus expected that we observe an overshoot in the central region.
The overshoot around \SI{30}{\milli\m} are most likely wall events on average reconstructed towards the center as well as real krypton events.

% r-plot for homogeneity here
% This plot is missing!!!
% TODO: plot aus notebook speichern und hier einfügen...
\begin{figure}
\centering
\includegraphics[width=0.95\textwidth]{Figures/th.jpeg}  % {Figures/r-homogen.png}
\caption[Kr r-histogram Homogeneity]{
        Sample Text.
        Sample Text..
        Sample Text...
    }
\label{fig:r-hist-homogen}
\end{figure}

Also, the projection onto the z-axis has to be checked to test homogeneity.
The respected histogram for z is shown in fig.~(TODO ref. z plot here).
We see stronger fluctuations within the errors.
However, the data is still in accordance with homogeneously distributed events inside the fiducial volume.
(Größe um die Aussage zu unterstützen?)
We can conclude that from a homogeneity point of view we can confirm an unbiased Krypton event selection.

% z-plot for homogeneity here
% This plot is missing!!!
% TODO: plot aus notebook speichern und hier einfügen...
\begin{figure}
\centering
\includegraphics[width=0.95\textwidth]{Figures/th.jpeg}  % {Figures/z-homogen.png}
\caption[Kr z-histogram Homogeneity]{
        Sample Text.
        Sample Text..
        Sample Text...
    }
\label{fig:z-hist-homogen}
\end{figure}



\paragraph{The decay time} of \isotope[83m1]{Kr} is another way of testing our selection.
The time difference $ t_\mathrm{diff} = t_{\mathrm{S}1_\mathrm{b}} - t_{\mathrm{S}1_\mathrm{a}} $ in an event represents\footnote{reflects?} its decay time.
Its distribution of incidence must hence follow a decaying exponential $ N \left( t \right) = N_0 \cdot \exp{\nicefrac{-t}{\tau_\mathrm{b}}} $, with $ \tau_\mathrm{b} = \SI{222.8}{\nano\s} $, the decay time of \isotope[83m1]{Kr}.
As we can see in fig.~(TODO ref decay time hist) a fit of the $ t_\mathrm{diff} $-hist yields an decay time parameter of \SI{}{\nano\s}. % TODO write number and error
This value is in accordance with the true decay time. % TODO in welcher accordance? präzieser wenn wert und fehler bekannt. cite paper? vllt eher bei der ersten Erwähnung
However, there are no events with decay times lower than \SI{240}{\nano\s} and the behaviour is only corretly modeled for $ t \ge \SI{330}{\nano\s} $.
The peak finder of strax is responsible for that, since it it not capable of splitting peaks that are closer than these \SI{240}{\nano\s}.
From this value up to \SI{330}{\nano\s} only some peaks can be split.
If the two S1 peaks can not be split up they are combined by strax to a single peak with larger width an combined area.
The further apart the two unseparated peaks, the larger the computed width.
This population of events with unseparated S1s is the larger widths arm of the \gls{s1a} population in fig.~(TODO ref auf main s1 area widths, oder wenn nicht zu sehen, diesen satz ändern oder löschen).
They do not survive this set of cuts.
Instead of tagging this population with another selection or tuning or replacing our peak finder we disregard these events.
The remaining Krypton events are, however, a representative population.
This selection does therefore not introduce a bias otherwise, because there is no correlation of decay time to any of the other dimensions.
By integrating our exponential fit from fig.~(TODO ref decay time hist) we can estimate that our Krypton selection has an acceptance of \SI{20}{\%}.
A peak finder capable of resolving closer peaks would be capable of increasing the acceptance drastically.

% TODO add waveforms of unseparated S1s (and compare to separated?)


\FloatBarrier


\newpage

\FloatBarrier
\subsection{Light Collection Efficiency}
\label{ssec:lce}
\FloatBarrier



% LCE definition
\paragraph{The \gls{lce} is defined as} the probability that a scintillation photon arrives at the photo-cathode of a \gls{pmt}.
This quantity is a function of the photons' initial position x,y, and z inside the \gls{tpc} with a second order dependence being the purity.
For a large number of initial photons, this corresponds to the fraction of all photons arriving at a photocathode over the total number of photons.
The \gls{lce} cannot be measured directly, because of intrinsic \gls{pmt} efficiencies - the \gls{qe} and \gls{ce}.
Therefore, when working with measured data, it is more convenient to work with a quantity including \gls{qe} and \gls{ce}.
To avoid ambiguities we define the \gls{de} as


\begin{equation}
    \mathit{DE}\left(\mathrm{x}, \mathrm{y}, \mathrm{z}\right) =  \mathit{LCE}\left(\mathrm{x}, \mathrm{y}, \mathrm{z}\right) \cdot \mathit{QE} \cdot \mathit{CE}
\end{equation}

where \gls{qe} and \gls{ce} vary for different \gls{pmt} models.
They can also vary rather significantly among \glspl{pmt} of the same model due to the fabrication process.

% contributors (not resolvable which contr. contributes how much to lce)
Different effects contribute to the collective quantity \gls{lce}.
The results of these effects on the \gls{lce} are not disjunct.
This means we cannot infer the contributions of a single effect from the measured \gls{lce} or rather \gls{de}.

\emph{Light attenuation} is the loss of photons per unit length traveled in material following an exponential decay.
In \gls{lxe}, the attenuation length $ \lambda_\mathrm{LXe} $ decreases with larger residual $ \mathrm{O}_2 $, $ \mathrm{H}_2\mathrm{O} $ and others.
An attenuation length of $ \lambda_\mathrm{LXe} > \SI{100}{\centi\m} $ has been measured in similar setups~\cite{Baldini05}.
The scattering length in \gls{xebra} is not experimentally determined but expected to be on the order of meters and therefore light attenuation in the liquid becomes negligible compared to losses on surfaces.

\emph{The reflectivity of \gls{ptfe}} surfaces is with around \SI{95}{\%} high for \gls{vuv} light.
The exact value depends on the quality of the surface treatment and no precise value for the dual-phase \gls{tpc} is known.
With special treatment, as done for \oneton, a reflectivity of \SI{99}{\%} can be achieved\cite{?}. % TODO cite resp. paper
Light is typically not only reflected once, but several times increasing the impact of the parameter.


\emph{The mesh's optical opacity} ($ O $) indicates which fraction of the light is lost on the transit through one of the electrode grids.
This value is well approximated by the geometrical coverage of the plane by the grid and thus makes up $ O \approx \SI{5}{\%} $.


\emph{The optical reflectivity at the \gls{lxe} surface} depends on refractive indices of gaseous and liquid Xenon and the incident angle of the light.
Reflected light travels further and thus has the tendency to be absorbed more likely due to, e.g., attenuation and mesh opacity.

\emph{The reflectivity of \gls{pmt} quartz window} reduces the light collected.
With the refraction index of $ n_\mathrm{quartz} = 1.56 $ the amount of light lost is larger in the gas phase than in the liquid phase.
As $ n_\mathrm{\gls{lxe}} \approx n_\mathrm{quartz} $ and $ n_\mathrm{\gls{gxe}} \approx 1 $, more photons are reflected at the quartz-windows of the top \glspl{pmt}, as they are surrounded by \gls{gxe}.


% z-dependence
\paragraph{The \emph{z} dependence} of the \gls{lce} is dominating the overall behavior.
With the meshes and liquid surfaces laying in the $ x\mbox{-}y $ plane, the respective effects have the most impact on the perpendicular $ z $-direction.
Additionally, the \gls{pmt} types, \gls{qe} and \gls{ce}, the surrounding media, and \gls{pmt} area coverage on the top side differ from the bottom side.
This increases disparity and contributes to the $ z $ trend shown in fig.~\ref{fig:ce_vs_z} which shows a linear behavior.


% CE vs z
\begin{figure}
\centering
\includegraphics[width=0.95\textwidth]{Figures/th.jpeg}  % {Figures/CE_vs_z.png}  % TODO make plot
\caption[Collection Efficiency in $ z $]{
        Collection Efficiency in $ z $
    }
\label{fig:ce_vs_z}
\end{figure}

% r-dependence
\paragraph{The \emph{r} dependence} of the \gls{lce} is minor and does not have a clear trend within its errors as shown in fig.~\ref{fig:ce_vs_r}.

% CE vs r
\begin{figure}
\centering
    \includegraphics[width=0.95\textwidth]{Figures/CE_vs_r.png}  % {Figures/th.jpeg}  % plot (notebook "Analysis/corrections/First area corrections")
\caption[Collection Efficiency in $ r $]{
    Collection Efficiency in $ r $
    }
\label{fig:ce_vs_r}
\end{figure}

% correction of S1-LCE
The correction of the \gls{lce} here means correcting for the position-dependent light losses based on Krypton calibration data as a first step.
Since some light is always lost, the maximal light measured does not equal the initial number of photons.
We thus correct on a relative scale.
The second step is converting to an absolute scale.
% This step is later done using simulation data from the \gls{nest} software~\cite{?}.  % TODO cite

In the first step, we only correct based on the $ z $-position.
The data for $ r $ shows ambiguities, which is thus not corrected.
In fig.~\ref{fig:s1area_vs_z}, we fit the S1 normalized to the mean $ \mathrm{S1_{a,rel}} = \mathrm{S1_a} \cdot \left( \overline{\mathrm{S1_a}} \right)^{-1} $ over $ z $.  % TODO Satz nochmal schreiben!!
Since we are working on a relative scale and due to how we switch to the absolute, it's justified to correct for the mean.
The S1 area (here S1), as well as the parameters from the linear fit, slope $ m_\mathrm{opt} $ and intercept $ c_\mathrm{opt} $, are used to calculate the \gls{cs1}:

\begin{equation}
    \mathrm{cS1}_\mathrm{a,b} = \frac{ \mathrm{S1} }{ z \cdot m_\mathrm{opt} + c_\mathrm{opt} }
\end{equation}

% rel S1 vs z
\begin{figure}
\centering
    \includegraphics[width=0.95\textwidth]{Figures/z-corr-fit.png}  % {Figures/th.jpeg}
\caption[Normalized S1 over $ z $ Fit]{
    Normalized S1 over $ z $ Fit
    }
\label{fig:s1area_vs_z}
\end{figure}



% area hist: compare csi to si





% vgl. Monte Carlo? gleich nach z-/r- dependence?





\newpage

\FloatBarrier
\subsection{Electron Lifetime}
\label{ssec:e-lifetime}
\FloatBarrier


Electrons have a certain probability to be captured by impurities like $ \mathrm{O}_2 $ per unit length drifted.
For a fixed drift velocity, this relates directly to a probability per unit time.
An initial quantity of $ N_0 $ electrons in an electron cloud at the point of interaction is reduced to
\begin{equation}
    N \left( t_\mathrm{drift} \right) = N_0 \cdot \mathrm{e}^{ \left( \nicefrac{-t_\mathrm{drift}}{\tau} \right)}
    \label{eq:e-lifetime}
\end{equation}
electrons in the gas gap.
Therefore, the S2 area declines with larger depth in the \gls{tpc} as shown in fig.~\ref{fig:s2-vs-t}.
$ \tau $ is called the \emph{electron lifetime}.
We can improve (larger $ \tau $) the electron lifetime by reducing the number of impurities inside the \gls{lxe} by cycling through the getter as described in % TODO Referenz auf getter/gassystem...: \ref{sec:}
Thereby, fewer electrons are lost while they drift.

Electron lifetime has a large impact on the quality of our data.
The fewer electrons are in an S2 signal, the more some fields' errors are dominated statistically, as the $ x-y$ position and the S2 area.
The relative width of the S2s' area distribution gets larger when more electrons are lost.
With low lifetimes below XX, electron-clouds originating towards the bottom of the \gls{tpc} can become so small that they are not recognized as S2s anymore.  % TODO add lifetime threshold  --> vllt mit einem plot wie von patrick elt vs. run-nr von kr runs --> dann auch plot und ref hinzufügen
These peaks can thus not be reconstructed into an event and are lost for most analysis purposes.
% lone hits!!
Impurities that capture an electron can be photoionized by light from an S1 or S2 releasing an electron as described in sec.~\ref{sec:se-source}.  % ref auf sec:single_electrons
With smaller electron lifetimes we thus also observe higher \gls{se} rates and therefore have a higher probability of coinciding \glspl{se}.
These mimic S2s from low energy interactions, e.g. the dark matter sector - another reason to invest effort into the purification of \gls{lxe} in large-scale experiments with ambitious science goals.

To correct for electron lifetime we fit the S2 area over $ t_\mathrm{drift} $ of a Krypton run with eq.~\ref{eq:e-lifetime}.
For every run with the same purity, we can calculate the lifetime corrected S2 area (here S2), $ \mathrm{S2_\mathrm{e_\tau c}} $, using the best-fit parameters $ \tau_\mathrm{opt} $ and $ N_0 $ with

\begin{equation}
    \mathrm{S2_\mathrm{e_\tau c}} = \frac{ \overline{\mathrm{S2_{Kr}}} \cdot \mathrm{S2} }{ N_0 \cdot \mathrm{e}^{ \left( \nicefrac{-t_\mathrm{drift}}{\tau_\mathrm{opt}} \right)} }.  %
\end{equation}

$ \overline{\mathrm{S2_{Kr}}} $ is the mean area of the S2s in the Krypton selection in the Krypton run the correction is based on.
As opposed to the number of electrons reaching the gas gap, we now have a measure of the number of electrons at the point of interaction.
For a set field configuration, the number of electrons created in an interaction of a certain energy is constant.
This gives a narrower distribution of the respective S2 areas as shown in fig.~\ref{fig:s2-area-hist-after-eltc} in the case of our monoenergetic Krypton source.


% S2 area hist after e-lifetime corr
\begin{figure}
    \centering
    \includegraphics[width=0.95\textwidth]{Figures/kr_s2_after_elt.png}  % {Figures/th.jpeg}
    \caption[S2 area e-lifetime corrected]{
        S2 area e-lifetime corrected
    }
    \label{fig:s2-area-hist-after-eltc}
\end{figure}



\newpage

\FloatBarrier
\subsection{scint-inhomo}
\label{ssec:scint-inhomo}
\FloatBarrier


While \gls{elt} is a vertical effect, there are two main effects altering the S2 area in the horizontal plane.
As shown in sec.~\ref{sec:xebra}, the light is produced in the gas gap by proportional scintillation.
Since the region is narrow and close to the top \gls{pmt}s, the \gls{lce} is higher directly under a \gls{pmt} and therefore describes a pattern in the x-y plane as shown in fig.~\ref{fig:lce-xy-sim-alex} from simulations.
In the simplest case of a radial symmetric behaviour, the effect can be described in one dimension.
As this is not the case here we use the x and y coordinate to describe the effect entirely.
The interaction depth, however, does not influence the \gls{lce}.

We compare the simulation prediction to our Krypton data in fig.~\ref{fig:S2-lce-x-y-krypton}.  % TODO add plot
The data does not follow the same trend as the simulation.
This can either be due to too simplistic simulations or we can not explain the trend just by \gls{lce}.
Since the data does not reflect the symmetry of the \gls{tpc} setting it is more likely that the discrepancy does not stem from \gls{lce}\footnote{symmetry gutes argument?}.

Beside \gls{lce}, the other effect that influences the light measured of an S2 signal are local differences in the scintillation process.
The number of photons produced per electron depends on the gas pressure, the voltage between gate and anode and the extent of the gas- as well as the liquid gap.
While the gas pressure and the voltage cannot change locally, the gap sizes can due to electrostatic sagging or buckling or mechanical stress.
A \emph{larger distance} between the two grids lower the field strength.
With a lower field, the electrons are less energetic and produce less light, on the one hand.
However, on the other hand the scintillation length is longer, raising the probability for scattering and thus scintillation.
This counteracts the lower light yield due the lower field and at the same time raising the peaks width due to the longer scintillation time of one electron.
A \emph{smaller distance} between the two grids raise the field strength.
The higher energetic electrons produce more light in scintillation and the peak has a smaller width due to the smaller gap.
The Krypton S2 peaks which also overlap are too broad to see the differences in width\footnote{müsste ich belegen. stimmt das überhaupt?}.
Formel oder so? siehe xenon note... könnte rein.

Although, the effect of \gls{lce} can not be distinguished from scintillation inhomogeneities by data, we can correct for both at the same time.
These corrections can then only be applied for runs with the same fields.
\gls{lce} is a detector parameter and does not change with different fields.
We could apply the correction for \gls{lce} to all runs.
For the scintillation, however, this is not the case.




\emph{Pure sagging or buckling} of the electrode grids can be described by a radial symmetric $ r^2 $ trend which describes the actual $ cosh\left( r \right) $ well with some advantage with regard to the computing power.
It is however possible that the center of the sagging does not coincide with the center of the \gls{tpc}.
In this case a transformation to radii with respect to another center or a description in two dimensions must be chosen.
The latter has the advantage of easy combination with other models of other effects like an additional tilt.
TODO hier Gleichung des Modells einfügen

\emph{Pure tilt} of the electrode grids is described by a two dimensional linear progression(wort ok?).
TODO hier Gleichung des Modells einfügen

Whenever we are unable to describe the trend of our data analytically as a fit to these models a purely data driven correction is to be chosen.
We take the mean of the S2 area per x-y-bin.
In each bin we want its center to be closer to each point inside the bin than to the surrounding centers.
Ideally, we would choose round bins then but they provide overlaps or areas that are not covered.
The optimal bin shape is a hexagon, for these two reasons.

% TODO:
TODO: verschiedene data driven models gegen die daten checken und beschreiben, worum es sich handeln kann.


% TODO:
Hier erklären, wie die correction vorgenommen wird? entweder fit oder direkt gebinnte daten als map speichern und beim laden interpolieren. denke hier ist die richtige stelle dafür. um es im strax kapitel zu erklären, hat man zu wenig hintergrund wissen..

When we choose to correct based on a fit to a certain model $ g\left( x,y \right) $ the correction is applied in the same manner as previously:

\begin{equation} % TODO formel nicht fertig, bisher nur kopiert aus kr sub-chap 4
    \mathrm{cS2} = \mathrm{S2_\mathrm{e_\tau c}} \cdot g\left( x,y; \mathbf{p} \right)^{-1}
    \label{eq:cs2}
\end{equation}

In the case of the purely \emph{data driven} model, however, the values and position of each bin is saved.
The correction value of a certain event is retrieved by interpolating between the values of the closest bins to the event position\footnote{sketch oder mathem. ausführen?}.



\newpage

\FloatBarrier
\subsection{scint-gain}
\label{ssec:scint-gain}
\FloatBarrier


To free a photon or electron as a primary scintillation quantum, an average energy of $ W = \SI{13.7}{\electronvolt} $ is required.
We, therefore, get a fixed number of quanta $ n = n_\gamma + n_\mathrm{e} $ in the scintillation process based on the total energy deposited in an interaction.
The ratio $ \nicefrac{n_\gamma}{n_\mathrm{e}} $ varies depending on the drift field strength and energy of the interacting particle.
We can extract the scintillation gains $ g_1 $ (primary) and $ g_2 $ (secondary) using the ratios at different energies by introducing $ n_\mathrm{\gamma} = \nicefrac{\mathrm{cS1}}{g_\mathrm{1}} $ and $ n_\mathrm{e} = \nicefrac{\mathrm{cS2}}{g_\mathrm{2}} $ to express

\begin{equation}
    E = W \left( n_\gamma + n_\mathrm{e} \right) = W \left( \frac{\mathrm{cS1}}{g_1} + \frac{\mathrm{cS2}}{g_2} \right).
    \label{eq:energy-correction}
\end{equation}

$ g_1 $ and $ g_2 $ are purely detector dependent quantities that are used to compare detectors to one another.
They measure how many \gls{pe} we detect per initial photon or electron, respectively.
In the case of $ g_2 $, the amplification of the proportional scintillation in the gas gap, as well as the associated extraction efficiency is also contained.
We are interested in the comparison to the hermetic \gls{tpc}, the single-phase \gls{tpc}, as well as large scale \glspl{tpc} like \nton.
When cS1 and cS2 are corrected relative to the mean or maximum of their measured areas or the theoretically expected absolute, the meaning and values of $ g_{1,2} $ do change.
Therefore, it is important to give this reference.
In this work, we correct the electron lifetime to the light of the interpolated number of initial electrons.
We correct all other effects relative to the mean of the used areas as this yields a more stable value.

To extract $ g_1 $ and $ g_2 $ from measurements, we reformulate eq.~\ref{eq:doke-fit} with $ \mathit{QY} = \frac{cS2}{E} $ and $ \mathit{LY} = \frac{cS1}{E} $ to

\begin{equation}
    \mathit{QL} = - \mathit{LY} \frac{g_1}{g_2} + \frac{g_2}{W}.
    \label{eq:doke-fit}
\end{equation}

By means of this correlation, we can fit the doke plot in fig.~\ref{fig:doke} linearly with $ m = - \nicefrac{g_2}{g_1} $ and $ c = \nicefrac{g_2}{W} $.  % TODO add plot
Although more than one mono energetic calibration source is necessary to use this method, we can use different field configurations to obtain a good measure of the correct scintillation gains.

% DOKE
\begin{figure}
\centering
\includegraphics[width=0.95\textwidth]{Figures/th.jpeg}  % {Figures/doke.png}  % TODO make plot
\caption[Doke plot]{
        Sample Text...
    }
\label{fig:doke}
\end{figure}

Another way to determine the correct $ g_1 $ and $ g_2 $ from a single source with one field is to use the same approach on an event basis, rather than on a run basis.
We fit an ellipse contour to the Krypton population in the $ \nicefrac{\mathrm{cS2}}{E} $-$ \nicefrac{\mathrm{cS1}}{E} $ space, fig.~\ref{fig:elipse}.

% ellipse fit
\begin{figure}
\centering
\includegraphics[width=0.95\textwidth]{Figures/th.jpeg}  % {Figures/ellipse.png}  % TODO make plot
\caption[Ellipse Fit]{
        Sample Text...
    }
\label{fig:ellipse}
\end{figure}




\newpage

\FloatBarrier
\subsection{eng-scale}
\label{ssec:eng-scale}
\FloatBarrier


We showed the correction of position dependent effects in sec.~\ref{ssec:},~\ref{ssec:}.  % TODO ref einfügen
From energy conservation considerations we got the scintillation gains in sec.~\ref{ssec:}.  % TODO ref
With the scintillation gains we can convert cS1 and cS2 to the interaction energy in \textit{eV}.
However, this conversion from $ \mathit{PE} $ is not generally valid.
The Krypton source we use (sec.~\ref{ssec:}) emits an predominantly electrons and also photons, which both interact with the Xenon's shell electrons, what we call \gls{er}. %  TODO ref
In the case of Neutrons and \gls{dm}, a \gls{nr} takes place with a different energy transfer.
A larger amount of energy is transfered to heat, compared to \gls{er}, which we cannot detect and is thus considered \emph{lost}.

%  TODO weiterschreiben... was genau??
TODO weiterschreiben





\newpage



\subsection{bg}
\label{ssec:bg}



\newpage
\FloatBarrier
