\FloatBarrier
\chapter{TPC Characterization}
\label{chap:Analysis}
\FloatBarrier

Analysis Chap.-Intro

\FloatBarrier
\section{Single Electrons}
\label{sec:SE}
\FloatBarrier

A \gls{se} has a characteristic signal signature.
We can assign a typical area and width.
This helps us understand S2 signals as they are composed of multiple electrons and the same rules apply for the amplification.
With the mean area of an \gls{se}, the amplification gain $ g_\mathrm{SE} $, we can count the electrons in an S2 signal\footnote{TODO keine extraction efficiency in g2 von SEs. können es hier amplification gain nennen und später bei krypton darauf hinweisen und extract. eff berechnen}.
Furthermore, the width and rate of \gls{se} are of interest and will be examined.
First, we discuss the origin of \gls{se} in the \gls{tpc} followed by the data selection criteria.



\FloatBarrier
\subsection{source}
\label{ssec:source}
\FloatBarrier


















%%%%%%% fig and eq

% Krypton Decay Scheme
\begin{figure}
    \centering
    \includegraphics[width=0.95\textwidth]{Figures/th.jpeg}  % {Figures/scheme_kr_decay.png}
    \caption[Decay Scheme Krypton]{Kr-decay scheme\cite{kr_scheme}.}
    \label{fig:test}
\end{figure}


\begin{equation}
    E = W \left( n_\gamma + n_\mathrm{e} \right) = W \left( \frac{\mathrm{cS1}}{g_1} + \frac{\mathrm{cS2}}{g_2} \right)
    \label{eq:test}
\end{equation}



\newpage

\FloatBarrier
\subsection{tagging}
\label{ssec:tagging}
\FloatBarrier


We discuss our data selection cuts and \gls{se} selection here.  Further reference to our \gls{se} population will be based on the selection discussed here.

\paragraph{After S2 selection:} Most \gls{se} signals are expected within one drift length after a large S2 peak.
We therefore first select larger S2 peaks and select every peak that comes within one drift length after such an S2.
In case another large S2 peak comes before that drift length window closes we do not select the S2 peak, but reopen a new window of the length of one drift length.

\paragraph{Two Channels Cut:} We also demand that at least two \gls{pmt}s contribute to the peak.
A peak from just one \gls{pmt} is very likely to be a dark count or noise.

\paragraph{\gls{aft} cut:} As S2 photons are produced in a narrow z-range close to the top \gls{pmt}s, they have one probability to be detected by the top rather than the bottom \gls{pmt}.
The result is a S2 specific \gls{aft}, $ \mathit{AFT}_\mathrm{S2} $.
We can see in fig.~\ref{fig:se-aft} that here $ \mathit{AFT}_\mathrm{S2} \approx 0.35 $.  % TODO: plot machen mit a+b/sqrt(area) und flat cuts: aft vs area
However, the lower the signal the more statistical fluctuation we have.
In the SE region we therefore have a wide spread of \gls{aft}s.
Cutting on the contours given by $ a + \nicefrac{b}{\sqrt{\mathit{area}}} $ does not improve our \gls{se} population significantly, but would help with larger S2s.
We get better results with a flat \gls{aft} cut $ \mathit{AFT}_\mathrm{min} = 0.05 $ and $ \mathit{AFT}_\mathrm{max} = 0.95 $ cutting bigger population clearly not belonging to \gls{se}s.

% SE AFT
\begin{figure}
    \centering
    \includegraphics[width=\figw\textwidth]{Figures/th.jpeg}  % {Figures/se_aft.png}
    \caption[AFT vs Area Single Electrons]{
        Single Electron AFT vs Area.
    }
    \label{fig:se-aft}
\end{figure}

% TODO weiter oben reinschreiben, welche areas wir erwarten
TODO AB HIER INS G2 KAPITEL?
The area histogram up to \SI{100}{\mathit{PE}} with all the previous cuts applied fig.~\ref{fig:se-area-hist} shows one broader population at just above \SI{20}{\mathit{PE}} with a long tail.
We identify this as our \gls{se} population where in the tail we have occurrences of multiple electrons simultaneously.
Another slightly overlapping population at areas smaller than \SI{10}{\mathit{PE}} is background.
Before fitting the population to extract the mean as the \emph{amplification gain} we want to make sure that the background population does not influence the result of the fit.
We can check this by reducing the background and compare the fit results.
A stable value hints towards low impact of the population on the fit.

% SE area hist
\begin{figure}
    \centering
    \includegraphics[width=\figw\textwidth]{Figures/th.jpeg}  % {Figures/se_area_hist.png}
    \caption[Histogram Area Single Electrons and Background]{
        Single Electron AFT vs Area.
    }
    \label{fig:se-area-hist}
\end{figure}


We choose the representative groups background, \gls{se}s and tail based on areas.
The representative regions are indicated in fig.~\ref{fig:se-area-hist} by the different background colors spanning from \numrange{0}{7} (background, green), \numrange{15}{30} (\gls{se}, red) and \numrange{40}{100}$ \,\mathit{PE} $ (tail, orange).
Furthermore, we compare the representatives in other dimensions to each other to find one where the background's is well separable from the other two.
The peak length, risetime and number of hits and a combination of these are possible dimensions for the cut.
The respective histograms and cuts are shown in fig.~\ref{fig:len-cut-se},~\ref{fig:len-cut-se},~\ref{fig:len-cut-se}.
We fit all the resulting populations in an area histogram with % the model $ f_\mathrm{SE}\left( \ar \right) $


    % f_\mathrm{SE}\left( \ar \right) =
\begin{align}
    &\sum_{n} G \left(  \ar; n \cdot g_\mathrm{SE}, \sqrt{n} \cdot \sigma_\mathrm{SE}, A_n \right) +
    G_\mathrm{bg} \left( \ar;  \mu_\mathrm{bg}, \sigma_\mathrm{bg}, A_\mathrm{bg} \right) \\
    &\mathrm{where~} G \left( x;  \mu, \sigma, A \right) = A \cdot \exp{ \frac{-\left( x - \mu \right)^2}{2\cdot\sigma^2}}
    \label{eq:se-fit-model}
\end{align}


is a Gaussian function, for n from \numrange{1}{4}.
The fits in fig.~\ref{fig:se-area-fits} all yield $ g_\mathrm{SE} $ that are compatible within their uncertainties.





\newpage

\FloatBarrier
\subsection{g2}
\label{ssec:g2}
\FloatBarrier

nicht sicher, wann genau ich den cut machen soll von tagging zu g2, deshalb ist bisher alles in tagging.

\newpage

\FloatBarrier
\subsection{width}
\label{ssec:width}
\FloatBarrier


Once extracted into the gas phase an electron has a certain probability to emit a photon per path length traveled until it reaches the anode. % TODO quelle angeben
With a given drift velocity, the time of possible emittance is hence fixed by the total path length.
This time interval is an upper limit for the length, or \SI{100}{\%} width of a \gls{se} peak, and evidently also other width quantiles.
As the width is very sensitive to the size of the gas gap, the width is a good indicator of deformations of the grid.
As discussed, a spacial mapping is not feasible.

% S2s usually have $ z $ dependent widths due to longitudinal diffusion.
% The arrival time at the liquid interface of the electrons in one S2 electron clound spreads dependent on the drift time.
Due to longitudinal diffusion, S2 widths depend, among others, on the drift time, or interaction depht $ z $.
The spread of the electron cloud increases accordingly with $ z $ when arriving at the liquid interface.
For \gls{se} this is not the case since these are single quanta.
\glspl{se}, however, tell us the intrinsic width of one electron which can be used to calculate the exact time between the first and last electron in the electron cloud.


Ohne eine Analyse sollte ich das vielleicht eher in ein gemeinsames kapitel mit der rate stecken und sowas wie outlook drauß machen?










\newpage

\FloatBarrier
\subsection{Single Electron Rates}
\label{ssec:rate}
\FloatBarrier


We calculate the rate of \glspl{se} by integrating the fit of the histogram in fig.~\ref{fig:se-area-hist}.
We integrate each Gaussian in eq.~\ref{eq:se-fit-model} individually, disregarding the background and weighting the others with $ n $.
In this example run, we find a \gls{se} rate of $ r_\mathrm{SE} = (310\pm30)\,\mathrm{Hz} $.

\glspl{se} or few-electron signals coinciding with S1s or S1 like signals can mimic \gls{dm} or other low energy signals.
An additional investigation of small S1-like signal rates can be conducted to infer a coincidence rate for events these mimic signals.
This investigation is however beyond the scope of this work.







\newpage

\FloatBarrier
\section{Krypton}
\label{sec:Kr}
\FloatBarrier

With an appropriate source, we can correct position-dependent signal loss and further convert to an absolute energy scale.
The source of choice is an excited Krypton state.
We need to prepare the recorded signal by tagging the data from the source to investigate signal losses due to \gls{lce} and \gls{elt}.
With the correction of these losses, we can convert to the absolute energy scale.
% depending of what else I write in the chapter, add more description here



\FloatBarrier
\subsection{\isotope[83][36]{Kr} Signal Production}
\label{ssec:source}
\FloatBarrier


The Rubidium isotope \isotope[83][37]{Rb} decays via two metastable excited states to the stable \isotope[83][36]{Kr} Krypton isotope.
It first transitions via electron capture $ \tau = \SI{86.2}{\day} $ to \isotope[83m]{Kr}.
This state has a half-life of \SI{1.83}{\hour} and decays via an energy emission of \SI{32.1}{\kilo\electronvolt}, see fig.~\ref{fig:scheme_kr}.
% Under the emission of an electron from internal conversion and subsequent emission of auger electrons and small amounts of X-rays its core is left as another excited state: \isotope[83m1]{Kr}.
It further decays to another excited state of \isotope[83]{Kr}.
The de-excitation to the \isotope[83]{Kr} ground state is mostly via internal conversion and with a small probability via gamma-rays.
The half-life of the latter is \SI{154}{\nano\s} and the decay energy is \SI{9.4}{\kilo\electronvolt}.


% Krypton Decay Scheme
\begin{figure}
    \centering
    \includegraphics[width=0.95\textwidth]{Figures/scheme_kr_decay.png}  % {Figures/th.jpeg}
    \caption[Decay Scheme Krypton]{
        Kr-decay scheme\cite{kr_scheme}.
    }
    \label{fig:scheme_kr}
\end{figure}

% want kr inside tpc
The solid Rubidium source is placed in a reservoir inside the gas system.
The Xenon flow can be directed through the reservoir or past it without contact.
When the flow follows the line that goes through the reservoir the gaseous \isotope[83m2]{Kr}, stemming from the Rubidium decay, is flushed along with the Xenon into the \gls{tpc}.
There, it disperses and distributes homogeneously within approximately a minute.
Once it decays, the second decay follows promptly, both resulting in \gls{lxe} scintillation, respectively, see sec.~\ref{sec:Xebra}.
Fig.~\ref{fig:waveform_kr} shows a sample waveform of this process.
With different drift field strengths the ratio of the scintillation quanta changes while the total number stays the same.
The stronger field separates the electrons from the respective ions more efficiently and thus supresses recombination.
Therefore, the light yield decreases with drift field strength while the charge yield increases.
Also the different decay branches of both Krypton decays - even though the energy is the same - result in slightly different light- and electron-responses in the scintillation process.
We are, however, not able to resolve the small differences in light- and charge yield caused by the different decay branches with our setup.
It will, however, contribute to the broadening of the energy resolution as predicted by the \gls{nest}~\cite{Szydagis13} model.
As the final decay product is \isotope[83]{Kr}, a stable Noble gas, it does neither react with other material nor contribute background radiation and can thus be left in the Xenon.
A conservatively estimated $ 2.5\cdot10^{6} $ atoms per filling and the Avogadro constant $ \mathrm{N}_\mathrm{A} \approx 6.02\cdot10^{23}\,\mathrm{mol}^{-1} $ yields a femtomol level after 1000 fillings.
At this level, the macroscopic properties of Xenon are not altered by the remaining Krypton.


% Kr sample waveform
\begin{figure}
    \centering
    \includegraphics[width=0.95\textwidth]{Figures/th.jpeg}  % {Figures/waveform_kr.png} TODO make this plot and save it to figs-dir
    \caption[Waveform Krypton]{
        sample waveform of krm2 to krm1 and krm1 to kr. s1, s2... etc.
    }
    \label{fig:waveform_kr}
\end{figure}


% Nest scint response
\begin{figure}
    \centering
    \includegraphics[width=0.95\textwidth]{Figures/nest_scint_response.png}  % {Figures/th.jpeg}
    \caption[Nest scint response]{
        Nest scint response.\cite{Szydagis13}
    } % TODO write subbing
    \label{fig:scint_response_nest}
\end{figure}


With the $ 2.5\cdot10^{6} $ initial radioactive nuclei of Krypton, after 7 half-lives less than 2500 nuclei that have not decayed are left.
This corresponds to a sub-percent level after less than \SI{13}{\hour}.
After a day we can thus easily take data without a noticeable amount of Krypton decays occurring.

% TODO the following has been written before the text above. it might need to be worked on again in order to not repeat to much.
% also the desired rate is calculated differently. needs to be changed.
\paragraph{A Krypton run} is a \gls{daq} time for which we fill the \gls{tpc} with the metastable Krypton Isotope \isotope[83m][36]{Kr} via the Rubidium source in the gas system as described above.
The amount of Krypton filled into the \gls{tpc} and thus the activity, cannot be controlled precisely.
This is however not necessary as the activity decreases quickly due to the short decay time.
The signals of two or more interactions at the same time - meaning an interaction happened before or just at the same time as the previous \st~is registered - cannot unambiguously be assigned to its correct physical counterpart.
As we have not investigated how efficient these events can be removed, we want to keep their occurrence small compared to the occurrences of single interactions.
We aim for a maximum of about $ 1\,\% $ of the events being in coincidence.


%
% \begin{equation}
%     P_{\lambda}\left(k\right) =  \frac{\lambda^k}{\exp^{-\lambda}}
% \end{equation}
%

One minute after starting the filling process of Krypton for a Krypton run it reaches the inner part of the \gls{tpc}.
We allow the Krypton to disperse in the liquid Xenon for five minutes to ensure that it is distributed homogeneously inside the \gls{tpc}.
The homogeneity of the distribution is shown in subsection~\ref{ssec:tagging}.



\FloatBarrier

\newpage

\FloatBarrier
\subsection{Event Selection}
\label{ssec:tagging}
\FloatBarrier



We can perform an energy calibration on the \gls{xebra} dual-phase \gls{tpc} using a homogeneously distributed, monoenergetic signal, as we have in Krypton runs.
The data of such a run has to be downselected by removing any signal that does not originate in the decay of the metastable Krypton we filled in.
Apart from Krypton decays, we find data from other background sources, dark counts, noise, and others that we will discuss.
The goal is to reduce any non-Krypton-signal contributions to a minimum while keeping as much signal as possible and without biasing the data by tagging a non-representative subset of the signal.
Although we will show that we can only tag a subset of our signal with a decay time of \SI{300}{\nano\s} and greater, this subset is representative in any other concern.
We will show that we can achieve a rejection of \dots and acceptance of \dots.  % TODO fill in rejection and acceptance
To illustrate the selection of data of this kind we choose run 250 as an example.
The cuts we discuss here are applied in the same way to any other Krypton run.


The mean time between the two consecutive Krypton decays is the lifetime $\tau_\mathrm{b} = \SI{222.8}{\nano\s}$.
Therefore, both decay signals are reconstructed into the same event.
Due to the ordering of peaks explained in sec.~\ref{sec:Strax} the \SI{32}{\kilo\eV} is the \gls{s1a}.
Following this convention, we will name the \gls{s1b}, the \gls{s2a}, and the \gls{s2b}, respectively.
We start our event selection by using the striking feature of the two consecutive decays in one event.


\paragraph{The two decays cut} firstly demands at least two S1 peaks in the event.
We cannot demand the same for S2 peaks, since \gls{s2a} and \gls{s2b} overlap in the vast majority of events, because of the greater intrinsic width of S2 signals and the short \gls{tpc} depth.
We thus regularly see the two S2 signals as one combined S2 peak and can only occasionally extract information about just one of them.
This is why we can secondly only demand a a minimum of one S2 instead of two.
A potential cut, therefore, is $n_\mathrm{peaks} \ge 3$ (sec.~\ref{sec:Strax}).
There are, however, two drawbacks compared to the cut we use instead.
First, we are not guaranteed two S1s and one S2, but just three peaks in total.
Theoretically, three S2 signals would satisfy the condition.
And second, a peak integrating to a negative or zero area can still count as a peak that contributes to $ n_\mathrm{peaks} $.
If on the other hand, no \gls{s1b} is present in an event its area will be set to zero.
Both drawbacks can be avoided by exploiting that non-present peaks have zero area.
The cut condition we use demands the areas of all, \gls{s1a}, \gls{s1b}, and \gls{s2a} to be strictly positive.

This cut rejects \SI{58.1}{\%} of \emph{all} events. % while the acceptance should be \SI{100}{\%} at first glance.
% However, as we will show, due to the current peak finder in strax, \gls{s1a} and \gls{s1b} are separable if they are at least \SI{300}{\nano\s} apart.
% Events with a distance lower than that are falsely rejected.
% We thus have an optimal overall acceptance of an estimated \SI{20}{\%}.
% The current cut does not account for all of the falsely rejected events.
It cuts all the events the weaker condition of $ n_\mathrm{peaks} \ge 3 $ cuts, while that one cuts only \SI{14.9}{\%}.


\paragraph{The time ordering cut} removes events in case they are not in agreement with the order of the decay chain of Krypton.
We base this cut on the time difference of \gls{s1a} and \gls{s1b}.
Since the index order is based on the area of the respective peak, we have to make sure the larger peak is detected first.
For this, we define the time difference between them as $t_\mathrm{diff} = t_{\mathrm{S}1_\mathrm{b}} - t_{\mathrm{S}1_\mathrm{a}}$.
We demand that $t_\mathrm{diff}$ is positive.
It is worth noting that $t_\mathrm{diff}$ follows the decay time statistics $\tau_\mathrm{b}$ of \isotope[83m1]{Kr}. % ('later' ok? sounds like temp ordering)

We purposely do not demand a proportion of $ \nicefrac{32}{9} $ between the areas of the two S1s - the proportion of their decay energies.
This condition requires stronger assumptions.
Therefore, we do not enforce this condition here but remove such misreconstruction later by other means.
The energy ratio hypothesis remains to be tested.

% Z-Cut
\begin{figure}[H]
    \centering
    \includegraphics[width=0.95\textwidth]{Figures/z-cut.png}  % {Figures/th.jpeg}
    \caption[Fiducial $ z $-cut]{
        $ z $-position histogram of all events in example Krypton run 250.
    An event that does not have an S1 or S2 peak gets assigned a drift time of zero, corresponding to $ z = 0 $.
    This explains the overshoot at this point.
    The second bump is at the gate position of $ z = \SI{-2.5}{mm}$ where we have wall-events.
    These events are suppressed by a fiducial cut $ z_\mathrm{max} = \SI{-8}{mm}$ indicated by the red dotted line on the right.
    The dotted red line on the left side indicates the lower $z$ cut $ z_\mathrm{min} = \SI{-68}{mm}$ preventing similar background contributions from the cathode grid,
    even though they are not as dominantly visible.
    The sudden drop-off below $ z = \SI{-70}{mm}$ indicates the maximum drift length of the \gls{tpc}.
    }
    \label{fig:fid-z-cut}
\end{figure}

\paragraph{The fiducial volume cut} rejects all events whose positions have been reconstructed outside a specific volume in the center of the drift volume.
% \emph{Fiducialization} is the self-shielding property of liquid Xenon.
In large-scale \glspl{tpc} like \nton, this cut is supposed to cut background $\alpha$ and $\beta$ decays from radioactive contamination of the electrodes or the PTFE reflectors, further referred to as \emph{wall-events}, as well as suppress low energetic $ \gamma $-events.
In experiments for rare events searches, these volumes are selected very carefully to minimize the background while achieving maximum exposure to reach higher statistics or alternatively stronger exclusion limits~\cite{?}.  % TODO cite
These fiducial volumes are typically described by higher-order polynomials based on Monte-Carlo simulations~\cite{?}. %TODO cite paper that shows fid vol cut with polynomials (xenon1t?)
In small \glspl{tpc} like the ones used in \gls{xebra} there is no benefit from this precise exclusion.
Our fiducial volume cut has also not to be optimized for high exposure, but rather for background exclusion and the accuracy limits of the position reconstruction.
This justifies a more simplistic modeling of the fiducial volume.
We define a cylindrical volume by a maximal and minimal value for the height $z$ and a maximal radius $r_\mathrm{max} = \SI{23}{\milli\m}$.
The latter bound is mentioned by the reliability of position reconstruction~\cite{ABism} and the effect on the complete Krypton selection is shown in fig.~\ref{fig:r-hist-homogen}.
The $z$ cut of $ z_\mathrm{min} = \SI{-68}{mm} $ and $ z_\mathrm{max} = \SI{-8}{mm} $ is chosen based on data, as shown in fig.~\ref{fig:fid-z-cut} and selects the region of linear drifttime-distance relation.


% Area-Width after fid.
\begin{figure}[H]
\centering
\includegraphics[width=0.95\textwidth]{Figures/oterhS1_area_width_after_fid.png}  % {Figures/th.jpeg}
\caption[Area-Width Histogram smaller S1 after Fid. Cut]{
    Area-Width Histogram smaller S1 after Fid. Cut
    }
\label{fig:other_s1_area_width}
\end{figure}


\paragraph{The false smaller S1 pairing cut} removes events with apparent misclassification of the \gls{s1b} peaks.
Besides the existence and time order, we did not define any restrictions based on the \gls{s1b} peak so far.
This allows the pairing of coincident S1s of any kind, as long as they have the correct time ordering.
In fig.~\ref{fig:other_s1_area_width} we see the second-largest S1 area-width histogram of events that satisfy all previous selection criteria.
Although, we expect a single population due to the mono-energetic decay of Krypton we see two separate populations in the plot.
The population in the bottom left of fig.~\ref{fig:other_s1_area_width} has both lower width and lower areas than the other.
The width (sec.~\ref{sec:Strax}) of about \SI{10}{\nano\s} corresponds to waveforms that have most of their area in one sample which is not what we expect from real S1s, as the scintillating decay has a longer lifetime as discussed in~\ref{sec:Strax}.
Also, the area of \numrange{3}{20}$\,\mathrm{PE}$ is distinctively less than the expected S1 by a \SI{9}{\kilo\electronvolt} energy deposition from electrons or photons.
Both statements disqualify this population which we thus exclude with the following cuts based on figs.~\ref{fig:other_s1_area_cut}~and~\ref{fig:other_s1_width_cut}.
With the slightly conservative cuts at $ \mathrm{area}_\mathrm{S1b} = 30\,\mathrm{PE}$ and $ \mathrm{width}_\mathrm{S1b} = \SI{24}{\nano\s} $ of the second-largest S1 we remove mostly mismatches.


% Area Hist Cut
\begin{figure}[h]
\centering
\includegraphics[width=0.95\textwidth]{Figures/oterhS1_area_hist_cut.png}  % {Figures/th.jpeg}
\caption[Other S1 Area Histogram Cut]{
        Other S1 Area Histogram Cut
    }
\label{fig:other_s1_area_cut}
\end{figure}


% Width Hist Cut
\begin{figure}
\centering
\includegraphics[width=0.95\textwidth]{Figures/oterhS1_width_hist_cut.png}  % {Figures/th.jpeg}
\caption[Other S1 Width Histogram Cut]{
        Other S1 Width Histogram Cut
    }
\label{fig:other_s1_width_cut}
\end{figure}


\paragraph{The false main S1 paring cut,} similar to the \emph{false smaller S1 pairing cut} restricts the \gls{s1a} peak to avoid misclassification.
In fig.~\ref{fig:main_s1_area_width}, there are two distinct populations.
The one to the left spans about \SI{400}{PE} and has a broader width distribution.
While the one on the right spans about \SI{9000}{PE} at higher areas and with a narrower widths distribution.
The decay of \isotope[83m2]{Kr} that we want to tag as \gls{s1a} is monoenergetic and does thus not comply with the larger spread of \SI{9000}{PE}.
We remove this population with a cut in area such that we will not bias the real Krypton S1 population in other runs that we use this cut on.
In accordance with fig.~\ref{fig:main_s1_area_width}, we place the cut at $ area_\mathrm{max} = \SI{600}{PE}$.

We assume that our selection is completed.
This is tested against the homogeneity premise and against the decay time of \isotope[83m1]{Kr}, $\tau_\mathrm{b} = \SI{222.8}{\nano\s}$.
The remaining events are further referred to as Krypton events.


% Area-Width main S1
\begin{figure}
\centering
\includegraphics[width=0.95\textwidth]{Figures/S1_area_width_after_fid.png}  % {Figures/th.jpeg}
\caption[Area-Width Histogram of main S1 after Fid. Cut]{
        Area-Width Histogram of main S1 after Fid. Cut
    }
\label{fig:main_s1_area_width}
\end{figure}


\paragraph{A homogeneous spatial-distribution} of Krypton inside the \gls{tpc} is expected.
With the tagging of Krypton via the discussed cuts we can test this hypothesis against the data to test our selection.
A non-removed population from external sources would here result in an non-uniform overshoot inside the \SI{23}{\milli\meter} radius region.
In the histogram of events over equal volume spaced r-bins, fig.~\ref{fig:r-hist-homogen}, we observe counts inside the fiducial $ r $-cut of $r_\mathrm{max} = \SI{23}{\milli\m}$ that agrees with homogeneity.
The bins with the smallest $ r $-values show higher counts and thus suggests a deviation from uniformity.
However, we know that the neural net - utilized for the position reconstruction - has a tendency of projecting the position inwards~\cite{ABism}.
It is thus expected that we observe an overshoot in the central region.
The overshoot around $ r = \SI{30}{\milli\m} $ is due to a know tendency from the neural net that projects the position towards that radius if the interaction originates somewhere outside $ r_\mathrm{max} $.

% r-plot for homogeneity
\begin{figure}
\centering
\includegraphics[width=0.95\textwidth]{Figures/kr_r-homogen.png}  % {Figures/th.jpeg}
\caption[Kr r-histogram Homogeneity]{
        Kr r-histogram Homogeneity
    }
\label{fig:r-hist-homogen}
\end{figure}

Also, the projection onto the z-axis has to be checked to test homogeneity.
The respective histogram for z is shown in fig.~\ref{fig:z-hist-homogen}.
We see stronger fluctuations within the statistical uncertainties.
However, the data is still in accordance with homogeneously distributed events inside the fiducial volume.
We can conclude that the data is in agreement with the assumption.

% z-plot for homogeneity
\begin{figure}
\centering
\includegraphics[width=0.95\textwidth]{Figures/kr_z-homogen.png}  % {Figures/th.jpeg}
\caption[Kr z-histogram Homogeneity]{
    Kr z-histogram Homogeneity
    }
\label{fig:z-hist-homogen}
\end{figure}



\paragraph{The decay time} of \isotope[83m1]{Kr} is another way of testing our selection.
The time difference $ t_\mathrm{diff} = t_{\mathrm{S}1_\mathrm{b}} - t_{\mathrm{S}1_\mathrm{a}} $ in a Krypton event follows its decay time.
A histogram of $ t_\mathrm{diff} $ must hence follow a decaying exponential $ N \left( t \right) = N_0 \cdot \exp{\nicefrac{-t}{\tau_\mathrm{b}}} $, with $ \tau_\mathrm{b} = \SI{222.8}{\nano\s} $, the lifetime of \isotope[83m1]{Kr}, corresponding to an halflife of $ T_{\nicefrac{1}{2}} = 154.4\,\mathrm{ns} $.
As we can see in fig.~\ref{fig:kr-decaytime} a fit of the histogram of $ t_\mathrm{diff} $ yields a decay time parameter of $ \tau_\mathrm{b} = (220 \pm 5)\,\mathrm{ns} $.
This value is in a one accordance within one standard deviation with the literature value of the lifetime~\cite{?}.  % cite same paper as the scheme?
However, there are no events with decay times lower than \SI{240}{\nano\s} and the behavior is only correctly modeled for $ t \ge \SI{330}{\nano\s} $.
The peak finder of strax is responsible for that since it is not capable of splitting peaks that are closer than the \SI{240}{\nano\s}.
From this value up to \SI{330}{\nano\s} only some peaks can be split.
If the two S1 peaks cannot be split, they are combined by strax to a single peak with larger width and combined area.
The further apart the two unseparated peaks, the larger is the computed width.
This population of events with unseparated S1s is the larger widths arm of the \gls{s1a} population in fig.~\footnote{TODO ref auf main s1 area widths, oder wenn nicht zu sehen, diesen satz ändern oder löschen}.
They do not survive this set of cuts.
Instead of tagging this population with another selection or tuning or replacing our peak finder, we disregard these events.
The remaining Krypton events are, however, a representative population.
This selection does therefore not introduce a bias otherwise, because there is no correlation of decay time to any of the other fields.
By integrating our exponential fit from fig.~\ref{fig:kr-decaytime} we can estimate that our Krypton selection has an acceptance of \SI{20}{\%}.
A peak finder capable of resolving closer peaks would be capable of increasing the acceptance drastically.


% decay time hist
\begin{figure}
\centering
\includegraphics[width=0.95\textwidth]{Figures/kr_decaytime.png}  % {Figures/th.jpeg}
\caption[Krypton Decay Time Histogram]{
    Krypton Decay Time Histogram
    }
\label{fig:kr-decaytime}
\end{figure}

% TODO add waveforms of unseparated S1s (and compare to separated?)


\FloatBarrier


\newpage



\subsection{lce}
\label{ssec:lce}



\newpage

\FloatBarrier
\subsection{Electron Lifetime}
\label{ssec:e-lifetime}
\FloatBarrier


Electrons have a certain probability to be captured by impurities like $ \mathrm{O}_2 $ per unit length drifted.
For a fixed drift velocity, this relates directly to a probability per unit time.
An initial quantity of $ N_0 $ electrons in an electron cloud at the point of interaction is reduced to
\begin{equation}
    N \left( t_\mathrm{drift} \right) = N_0 \cdot \mathrm{e}^{ \left( \nicefrac{-t_\mathrm{drift}}{\tau} \right)}
    \label{eq:e-lifetime}
\end{equation}
electrons in the gas gap.
Therefore, the S2 area declines with larger depth in the \gls{tpc} as shown in fig.~\ref{fig:s2-vs-t}.
$ \tau $ is called the \emph{electron lifetime}.
We can improve (larger $ \tau $) the electron lifetime by reducing the number of impurities inside the \gls{lxe} by cycling through the getter as described in sec.~\ref{sec:Xebra}. % TODO Referenz auf getter/gassystem. DONE, aber falls die kapitel unterteilung nochmal verfeiernt wird, muss ich auch hier genauer refferenzieren.
Thereby, fewer electrons are lost while they drift.


% S2 area vs t
\begin{figure}
    \centering
    \includegraphics[width=0.95\textwidth]{Figures/s2-vs-t.png}  % {Figures/th.jpeg}
    \caption[S2 over Drifttime Fit]{
        S2 over Drifttime Fit
    }
    \label{fig:s2-vs-t}
\end{figure}


Electron lifetime has a large impact on the quality of our data.
The fewer electrons are in an S2 signal, the more some fields' errors are dominated statistically, as the $ x-y$ position and the S2 area.
The relative width of the S2s' area distribution gets larger when more electrons are lost.
With low lifetimes below XX, electron-clouds originating towards the bottom of the \gls{tpc} can become so small that they are not recognized as S2s anymore.  % TODO add lifetime threshold  --> vllt mit einem plot wie von patrick elt vs. run-nr von kr runs --> dann auch plot und ref hinzufügen
These peaks can thus not be reconstructed into an event and are lost for most analysis purposes.
% lone hits!!
Impurities that capture an electron can be photoionized by light from an S1 or S2 releasing an electron as described in sec.~\ref{ssec:se-source}.  % ref auf sec:single_electrons
With smaller electron lifetimes we thus also observe higher \gls{se} rates and therefore have a higher probability of coinciding \glspl{se}.
These mimic S2s from low energy interactions, e.g. the dark matter sector - another reason to invest effort into the purification of \gls{lxe} in large-scale experiments with ambitious science goals.

To correct for electron lifetime we fit the S2 area over $ t_\mathrm{drift} $ of a Krypton run with eq.~\ref{eq:e-lifetime}.
For every run with the same purity, we can calculate the lifetime corrected S2 area (here S2), $ \mathrm{S2_\mathrm{e_\tau c}} $, using the best-fit parameters $ \tau_\mathrm{opt} $ and $ N_0 $ with

\begin{equation}
    \mathrm{S2_\mathrm{e_\tau c}} = \frac{ \overline{\mathrm{S2_{Kr}}} \cdot \mathrm{S2} }{ N_0 \cdot \mathrm{e}^{ \left( \nicefrac{-t_\mathrm{drift}}{\tau_\mathrm{opt}} \right)} }.  %
\end{equation}

$ \overline{\mathrm{S2_{Kr}}} $ is the mean area of the S2s in the Krypton selection in the Krypton run the correction is based on.
As opposed to the number of electrons reaching the gas gap, we now have a measure of the number of electrons at the point of interaction.
For a set field configuration, the number of electrons created in an interaction of a certain energy is constant.
This gives a narrower distribution of the respective S2 areas as shown in fig.~\ref{fig:s2-area-hist-after-eltc} in the case of our monoenergetic Krypton source.


% S2 area hist after e-lifetime corr
\begin{figure}
    \centering
    \includegraphics[width=0.95\textwidth]{Figures/kr_s2_after_elt.png}  % {Figures/th.jpeg}
    \caption[S2 area e-lifetime corrected]{
        S2 area e-lifetime corrected
    }
    \label{fig:s2-area-hist-after-eltc}
\end{figure}



\newpage

\FloatBarrier
\subsection{scint-inhomo}
\label{ssec:scint-inhomo}
\FloatBarrier


While \gls{elt} is a vertical effect, there are two main effects altering the S2 area in the horizontal plane.
As shown in sec.~\ref{sec:xebra}, the light is produced in the gas gap by proportional scintillation.
Since the region is narrow and close to the top \gls{pmt}s, the \gls{lce} is higher directly under a \gls{pmt} and therefore describes a pattern in the x-y plane as shown in fig.~\ref{fig:lce-xy-sim-alex} from simulations.
The interaction depth, however, does not influence the \gls{lce}.

We compare the simulation prediction to our Krypton data in fig.~\ref{fig:S2-lce-x-y-krypton}.  % TODO add plot
The data does not follow the same trend as the simulation.
This can either be due to too simplistic simulations or we can not explain the trend just by \gls{lce}.
Since the data does not reflect the symmetry of the \gls{tpc} setting it is more likely that the discrepancy does not stem from \gls{lce}\footnote{symmetry gutes argument?}.

Beside \gls{lce}, the other effect that influences the light measured of an S2 signal are local differences in the scintillation process.
The number of photons produced per electron depends on the gas pressure, the voltage between gate and anode and the extent of the gas- as well as the liquid gap.
While the gas pressure and the voltage cannot change locally, the gap sizes can due to electrostatic sagging or buckling or mechanical stress.
A \emph{larger distance} between the two grids lower the field strength.
With a lower field, the electrons are less energetic and produce less light, on the one hand.
However, on the other hand the scintillation length is longer, raising the probability for scattering and thus scintillation.
This counteracts the lower light yield due the lower field and at the same time raising the peaks width due to the longer scintillation time of one electron.
A \emph{smaller distance} between the two grids raise the field strength.
The higher energetic electrons produce more light in scintillation and the peak has a smaller width due to the smaller gap.
The Krypton S2 peaks which also overlap are too broad to see the differences in width\footnote{müsste ich belegen. stimmt das überhaupt?}.
Formel oder so? siehe xenon note... könnte rein.

Although, the effect of \gls{lce} can not be distinguished from scintillation inhomogeneities by data, we can correct for both at the same time.
These corrections can then only be applied for runs with the same fields.
\gls{lce} is a detector parameter and does not change with different fields.
We could apply the correction for \gls{lce} to all runs.
For the scintillation, however, this is not the case.



% TODO:
TODO: verschiedene data driven models gegen die daten checken und beschreiben, worum es sich handeln kann. warum nicht vereinbar mit einfacher r-dependence?





% TODO:
Hier die erklären, wie die correction vorgenommen wird? entweder fit oder direkt gebinnte daten als map speichern und beim laden interpolieren. denke hier ist die richtige stelle dafür. um es im strax kapitel zu erklären, hat man zu wenig hintergrund wissen..



\newpage

\FloatBarrier
\subsection{scint-gain}
\label{ssec:scint-gain}
\FloatBarrier




\newpage



\subsection{eng-scale}
\label{ssec:eng-scale}



\newpage



\subsection{bg}
\label{ssec:bg}



\newpage
\FloatBarrier
