
\FloatBarrier
\subsection{Energy Calibration}
\label{ssec:eng-scale}
\FloatBarrier


We show the correction of position-dependent effects in sec.~\ref{ssec:lce}~and~\ref{ssec:e-lifetime}.  % TODO ref einfügen
From energy conservation considerations we got the scintillation gains in sec.~\ref{ssec:scint-gain}.  % TODO ref
With the scintillation gains, we can convert cS1 and cS2 to the interaction energy in \textit{eV}.
However, this conversion from \gls{pe} is not generally valid.
The Krypton source we use (sec.~\ref{ssec:source}) predominantly emits electrons and photons, which both interact with the Xenon's shell electrons, which we call \gls{er}. %  TODO ref
In the case of Neutrons and \gls{dm}, an \gls{nr} takes place with a different energy transfer.
A larger amount of energy is transferred to heat, compared to \gls{er}, which we cannot detect and is thus considered \emph{lost}.

%  TODO weiterschreiben... was genau??
TODO weiterschreiben




