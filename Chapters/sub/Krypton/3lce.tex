
\FloatBarrier
\subsection{Light Collection Efficiency}
\label{ssec:lce}
\FloatBarrier



% LCE definition
\paragraph{The \gls{lce} is defined as} the probability of a photon to arrive at a photo cathode of an \gls{pmt}.
This quantity is a function of the photons initial position x,y and z inside the \gls{tpc}.
For many initial photons this is all photons arriving at a photo cathode over the total number of photons.
The \gls{lce} can thus not be measured directly, because of intrinsic \gls{pmt} efficiencies, the \gls{qe} and \gls{ce}.
Therefore, when working with measured data it is more convenient to work with a quantity including \gls{qe} and \gls{ce}.
For simplicity and to avoid ambigouities we define the \gls{de} as


\begin{equation}
    DE\left(\mathrm{x}, \mathrm{y}, \mathrm{y}\right) =  LCE\left(\mathrm{x}, \mathrm{y}, \mathrm{y}\right) \cdot QE \cdot CE
\end{equation}

where are vary for different \gls{pmt} types.
They can also vary rather significantly with among \gls{pmt}s of the same type.

% contributors (not resolvable which contr. contributes how much to lce)
\paragraph{There are different effects} contributing to the collective quantity \gls{lce}.
The results of these effects on the \gls{lce} are not disjunct.
This means we cannot infer the contributions of a single effect from the measured \gls{lce} or rather \gls{de}.

\emph{Light attenuation v1} is the loss of photons per unit length traveled in material following an exponential decay.
In \gls{lxe}, the attenuation length $ \lambda_\mathrm{LXe} $ decreases with larger residual $ \mathrm{O}_2 $ and $ \mathrm{H}_2\mathrm{O} $.
The attenuation length of $ \lambda_\mathrm{LXe} = \SI{50}{\m} $ was measured in \oneton\cite{}.  % TODO: find paper where this was measured. Fabian told me this number. Was it in 1T?
\footnote{[TODO: Check where these 50m come from because alex assumed 100cm which is from a xenon papter. worse values have been measured before. i've we are arround 10cm, we'd loose over 50\% of the light within one(!!) drift length. this is not negligible anymore, compared to the ~1\% loss we assumed before.]}

\emph{Light attenuation v2} is the loss of photons per unit length traveled in material following an exponential decay.
In \gls{lxe}, the attenuation length $ \lambda_\mathrm{LXe} $ decreases with larger residual $ \mathrm{O}_2 $ and $ \mathrm{H}_2\mathrm{O} $.
An attenuation length of $ \lambda_\mathrm{LXe} > \SI{100}{\centi\m} $ has been measured in similar setups\cite{}.  % TODO: Cite! (A.Baldini at al., NIM A 545 (2005))
Assuming $ \lambda_\mathrm{LXe} = \SI{100}{\centi\m} $, \SI{25}{\%} of the light is lost over \SI{30}{\centi\m}, corresponding to approximately 4 drift lengths.

% in light atten reinpacken
Where the scattering length is not experimentally determant but expected to be on the oder of meters and therefore light atten. in the liq becomes neglict. compared to losses on surfaces.


\emph{The reflectivity of \gls{ptfe}} surfaces, here mainly the \gls{ptfe} reflectors, is with arround \SI{98}{\%} high for light with the \gls{lxe} scintillation wavelength.
The exact value depends on the quality of the surface treatment.
With special treatment as in \oneton a reflectivity of \SI{99}{\%} can be achieved\cite{}. % TODO cite resp. paper
Light is typically not only reflected once, but several times increasing the impact of the parameter.


\emph{The mesh's optical opacity} ($ O $) indicates which fraction of the light is lost on the transit through one of the electrode grids.
This value is well approximated by the geometrical coverage of the plane by the grid and thus computes to $ O = \SI{5}{\%} $.


\emph{The optical reflectivity at the \gls{lxe} surface} depends on refractive indices of gaseous and liquid Xenon, the incident angle of the light.
Reflected light, on the one hand, travels further until it hits the \gls{pmt} cathode when the light comes from the drift field region and eventually arrives at the bottom \gls{pmt}.
On the other hand, it travels less far originating in the gaseous xenon and is reflected back to the top array.
Greater travel distances have the trend to have more light absorbed from attenuation and mesh opacity.
With $ n_1 $ and $ n_2 $ the refraction indices and $ \alpha $ and $ \beta $ the angle of incident and reflection as shown in fig.~% \ref{fig:phase_transion_sketch}.  % TODO sketch einfügen
the reflectivity is given by
\begin{alignat}{2}
    \label{eq:reflectivityA}
    R_\perp &= \left( \frac{ n_1 \cos{\alpha} - n_2 \cos{\beta} }{ n_1 \cos{\alpha} + n_2 \cos{\beta} } \right)^2 &
            &=  \left( \frac{ \sin{\left( \alpha-\beta \right)}}{ \sin{\left( \alpha+\beta \right)} } \right)^2 \\
    R_\parallel &= \left( \frac{ n_2 \cos{\alpha} - n_1 \cos{\beta} }{ n_2 \cos{\alpha} + n_1 \cos{\beta} } \right)^2 &
                &=  \left( \frac{ \tan{\left( \alpha-\beta \right)}}{ \tan{\left( \alpha+\beta \right)} } \right)^2,
    \label{eq:reflectivityB}
\end{alignat}
where $ \beta $ is given via Snellius as $ \beta = \arcsin{\nicefrac{n_1}{n_2} \sin{\alpha} } $.
And the refraction indices $ n_1 $, $ n_2 $ are $ n_\mathrm{LXe} = XX \approx 2 $ and $ n_\mathrm{GXe} = XX \approx 1 $, depending on the light's direction through the surface.


\emph{The reflectivity of \gls{pmt} quartz window} reduces the light collected. With the refraction index of $ n_\mathrm{quartz} = 1.56 $ the amount of light lost is larger in the gas phase than in the liquid phase. The reflectivity is calculated with eq.~(\ref{eq:reflectivityA}, \ref{eq:reflectivityB}) depending on the incident angle and the materials refraction indices.


% z-dependence
\paragraph{The \emph{z} dependence} of the \gls{LCE} is the dominating the overall behaviour.
With the meshes and liquid surfaces laying in the $ x-y $ plane the respective effects have most impact on the perpendicular $ z $-direction.
Additionally, the \gls{pmt} types, \gls{qe} and \gls{ce}, surrounding media and \gls{pmt} area coverage on the top side differ from the bottom side.




% r-dependence


% correction of S1-LCE


% area hist: compare csi to si


% vgl. Monte Carlo?




