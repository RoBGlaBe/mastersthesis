
\FloatBarrier
\subsection{tagging}
\label{ssec:tagging}
\FloatBarrier


For work with Krypton data we record dedicated Krypton runs.
In a Krypton run we fill the \gls{tpc} with a certain amount of Krypton so that the rate is low enough to distinguish single interactions.
One minute after starting the filling process of Krypton it reaches the inner part of the \gls{tpc}.
After a total of five minutes we assume that the Krypton dispersed homogeneously within the \gls{tpc}.
We can confirm this assumption by observation.
Our Krypton runs are three minute recordings.


All following Krypton analyses require a selection of Krypton events to remove other interactions, dark counts, noise and misreconstructed events.
A clean Krypton sample is one that is dominated by Krypton without biasing the selection by favouring one certain subpopulation.
We will introduce a couple of cuts to accomplish this.


Although the two Krypton decays are separate interactions, they are reconstructed into the same event as the second decay has such a small decay time.
This means that we expect the largest S1 in an event to belong to the \SI{32}{\kilo\eV} decay, whereas the second largest S1 belongs to the \SI{9}{\kilo\eV}.
However, the respective S2 signals are so broad that their waveforms overlap in almost all cases.
We thus see the two S2 signals as one combined S2 peak and cannot extract information about just one of them.
This can directly be translated in our first necessary cut condition.
We demand that the main and second largest S1, as well as the main S2 have an area greater than zero.
On the one hand, this removes all events that don't have the respective peak, e.g. no S2 peak.
On the other hand, this also removes events where the respective peak is present, but its waveform has such a shape that it integrates to a non-positive area.
The second part of this condition is to demand that the larger S1 occurs before the other one.
For this we define the time difference between them as $t_{diff} = t_{S1_2} - t_{S1_1}$.
We demand that $t_{diff}$ is positive.
It is worth noting that $t_{diff}$ is the decay time of the event which we will use later again.
We purposely do not demand a proportion of $ \nicefrac{32}{9} $ between the areas of the two S1s.
This condition requires stronger assumptions.
Therefore, we do not want to enforce this condition here, but remove such misreconstructions later otherwise.


We chose a defined part of the central drift volume as the fiducial volume and only allow events in this region for our analyses.
\emph{Fiducialization} is the self shielding property of liquid Xenon.
Almost all background events from \gls{tpc} walls and the electrode meshes are thereby excluded.
In large scale \gls{tpc} with science goals these areas are selected very carefully minimize the background while achieving maximum exposure to reach higher statistics or alternatively stronger exclusion limits.
These fiducial volumes are typically described by higher order polynomials on the basis of Monte-Carlo simulations.
This kind of effort cannot be justified for the goals of this analysis.
Therefore, the description is not optimized for high exposure, but only for background exclusion.
At the same time, this justifies an easier modeling of the fiducial volume.
We define a cylindrical volume by a maximal and minimal value for the height $z$ and a maximal radius $r = \SI{23}{\milli\m}$.
While the $z$ cut of $z_{min}= \SI{-68}{mm}$ and $z_{max}= \SI{-8}{mm}$ is chosen based on data as shown in fig.~\ref{fig:fid-z-cut}  and has only the goal of just mentioned background exclusion, we have further restrictions for the radius.
The position reconstructed by the neural net is only accurate within this radial limit \cite{ABism}.


% Z-Cut
\begin{figure}
\centering
\includegraphics[width=0.95\textwidth]{Figures/th.jpeg}  % {Figures/z-cut.png}
    \caption[Fiducial z-cut]{Histogram of the event z positions of all events of an example Krypton run.
                             All events missing either S1s or S2s effectivly reconstruct to a z-position of zero explaining the high overshoot at this position.
                             The second bump is at the gate position of $ z = \SI{-2.5}{mm}$ and background events originating there.
                             These events are supressed by a fiducial cut at $ z = \SI{-8}{mm}$ indicated by the red dotted line on the right.
                             The dotted red line on the left side indicates the lower $z$ cut at $ z = \SI{-68}{mm}$ preventing similar background contributions from the cathode grid,
                             even though they are not as dominantly visible.
                             The sudden drop-off at below $ z = \SI{-70}{mm}$ indicates the maximum drift length of the \gls{tpc}.
                             }
\label{fig:fid-z-cut}
\end{figure}


% Area-Width after fid.
\begin{figure}
\centering
\includegraphics[width=0.95\textwidth]{Figures/th.jpeg}  % {Figures/oterhS1_area_width_after_fid.png}
\caption[Area-Width Histogram after Fid. Cut]{
        Sample Text.
        Sample Text..
        Sample Text...
    }
\label{fig:other_s1_area_width}
\end{figure}

In fig.~\ref{fig:other_s1_area_width} we see the second largest S1 area-width histogram of events that satisfy all previous selection criteria.
Although, we expect a single population due to the monoenergetic decay of Krypton we see two separate populations in the plot.
One population has both lower width and lower areas than the other.
The $50\,\%$ central area quantile of about \SI{10}{\nano\s} corresponds to waveforms which have most of their area in one sample which is less then what we expect from real S1s.
Also, the area of to \numrange{3}{20} is less than what we expect from simulations.
Both statements disqualify the just mentioned population which we thus exclude with the following cuts based on figs.~\ref{fig:other_s1_area_cut},~\ref{fig:other_s1_width_cut}.
With the slightly conservative cuts at $30\,\mathrm{PE}$ area and \SI{24}{\nano\s} width of the second largest S1 we remove most mismatches.
To further clean up the Krypton selection we investigate other dimensions (better word?).


% Area Hist Cut
\begin{figure}
\centering
\includegraphics[width=0.95\textwidth]{Figures/th.jpeg}  % {Figures/oterhS1_area_hist_cut.png}
\caption[Other S1 Area Histogram Cut]{
        Sample Text.
        Sample Text..
        Sample Text...
    }
\label{fig:other_s1_area_cut}
\end{figure}


% Width Hist Cut
\begin{figure}
\centering
\includegraphics[width=0.95\textwidth]{Figures/th.jpeg}  % {Figures/oterhS1_width_hist_cut.png}
\caption[Other S1 Width Histogram Cut]{
        Sample Text.
        Sample Text..
        Sample Text...
    }
\label{fig:other_s1_width_cut}
\end{figure}




\FloatBarrier





