
\FloatBarrier
\subsection{e-lifetime}
\label{ssec:e-lifetime}
\FloatBarrier


Electrons have a certain probability to be captured by impurities like $ \mathrm{O}_2 $ per unit length drifted.
For a fixed drift velocity this relates directly to a probability per unit time.
An initial quantity of $ N_0 $ electrons in an electron cloud at the point of interaction is reduced to
\begin{equation}
    N_{t_\mathrm{drift}} = N_0 \cdot \mathrm{e}^{ \left( \nicefrac{-t_\mathrm{drift}}{\tau} \right)}
    \label{eq:e-lifetime}
\end{equation}
electrons in the gas gap.
Therefore, the S2 area declines with greater depth in the \gls{tpc} as shown in fig.~\ref{fig:s2-vs-t}.
$ \tau $ is called the \emph{electron lifetime}.
We can improve (greater $ \tau $) the electron lifetime by reducing the amount of impurities inside the \gls{lxe} by cycling through the getter as described in % TODO Referenz auf getter/gassystem...: \ref{sec:}

Electron lifetime has a large impact on the quality of our data.
The fewer electrons are in a S2 signal the more we are dominated by statistical effects raising the relative width of any interaction investigated.
With low lifetimes below XX electrons clouds originating towards the bottom of the \gls{tpc} can become so small that they are not recognized as S2s anymore.  % TODO add lifetime threshold  --> vllt mit einem plot wie von patrick elt vs. run-nr von kr runs --> dann auch plot und ref hinzufügen
These peaks can thus not be reconstructed into an event and are lost for most analysis purposes.
% lone hits!!
Impurities that capture an electron can be photoionized by light from an S1 or S2 releasing an electron as described in SINGLE ELECTRONS.  % ref auf sec:single_electrons
With smaller electron lifetimes we thus also observe higher \gls{se} rates and therefore have a higher probability of coinciding \gls{se}s.
These mimic S2s from low energy interactions, e.g. the dark matter sector - another reason to invest effort into purification of \gls{lxe} in large scale experiments with ambitious science goals.

To correct for electron lifetime we fit the S2 area over $ t $ of a Krypton run with eq.~\ref{eq:e-lifetime}.
For every run with the same purity we can calculate the lifetime corrected S2, $ \mathrm{S2_\mathrm{e_\tau c}} $, using the best fit parameters $ \tau_\mathrm{opt} $ and $ N_0 $ with

\begin{equation}
    \mathrm{S2_\mathrm{e_\tau c}} = \frac{ \overline{\mathrm{S2_{Kr}}} \cdot \mathrm{S2} }{ N_0 \cdot \mathrm{e}^{ \left( \nicefrac{-t_\mathrm{drift}}{\tau_\mathrm{opt}} \right)} }.  %
\end{equation}

As apposed to the number of electrons reaching the gas gap we now have a measure of the number of electrons at the point of interaction.
For a set field configuration the number of electrons created in an interaction of a certain energy is constant.
This gives a narrower distribution of the respective S2 areas as shown in fig.~\ref{fig:s2-area-hist-after-eltc} in the case of our monoenergetic Krypton source.


% S2 area hist after e-lifetime corr
\begin{figure}
    \centering
    \includegraphics[width=0.95\textwidth]{Figures/th.jpeg}  % TODO bild einfügen
    \caption[S2 area]{
        Bla bla bla.
    }
    \label{fig:s2-area-hist-after-eltc}
\end{figure}


