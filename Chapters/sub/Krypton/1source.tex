
\FloatBarrier
\subsection{\isotope[83][36]{Kr} Signal Production}
\label{ssec:source}
\FloatBarrier


The Rubidium isotope \isotope[83][37]{Rb} decays via two metastable excited states to the stable \isotope[83][36]{Kr} Krypton isotope.
It first transitions via electron capture $ \tau = \SI{86.2}{\day} $ to \isotope[83m]{Kr}.
This state has a half-life of \SI{1.83}{\hour} and decays via an energy emission of \SI{32.1}{\kilo\electronvolt}, see fig.~\ref{fig:scheme_kr}.
% Under the emission of an electron from internal conversion and subsequent emission of auger electrons and small amounts of X-rays its core is left as another excited state: \isotope[83m1]{Kr}.
It further decays to another excited state of \isotope[83]{Kr}.
The de-excitation to the \isotope[83]{Kr} ground state is mostly via internal conversion and with a small probability via gamma-rays.
The half-life of the latter is \SI{154}{\nano\s} and the decay energy is \SI{9.4}{\kilo\electronvolt}.


% Krypton Decay Scheme
\begin{figure}
    \centering
    \includegraphics[width=0.95\textwidth]{Figures/scheme_kr_decay.png}  % {Figures/th.jpeg}
    \caption[Decay Scheme Krypton]{
        Kr-decay scheme\cite{kr_scheme}.
    }
    \label{fig:scheme_kr}
\end{figure}

% want kr inside tpc
The solid Rubidium source is placed in a reservoir inside the gas system.
The Xenon flow can be directed through the reservoir or past it without contact.
When the flow follows the line that goes through the reservoir the gaseous \isotope[83m2]{Kr}, stemming from the Rubidium decay, is flushed along with the Xenon into the \gls{tpc}.
There, it disperses and distributes homogeneously within approximately a minute.
Once it decays, the second decay follows promptly, both resulting in \gls{lxe} scintillation, respectively, see sec.~\ref{sec:Xebra}.
Fig.~\ref{fig:waveform_kr} shows a sample waveform of this process.
With different drift field strengths the ratio of the scintillation quanta changes while the total number stays the same.
The stronger field separates the electrons from the respective ions more efficiently and thus supresses recombination.
Therefore, the light yield decreases with drift field strength while the charge yield increases.
Also the different decay branches of both Krypton decays - even though the energy is the same - result in slightly different light- and electron-responses in the scintillation process.
We are, however, not able to resolve the small differences in light- and charge yield caused by the different decay branches with our setup.
It will, however, contribute to the broadening of the energy resolution as predicted by the \gls{nest}~\cite{Szydagis13} model.
As the final decay product is \isotope[83]{Kr}, a stable Noble gas, it does neither react with other material nor contribute background radiation and can thus be left in the Xenon.
A conservatively estimated $ 2.5\cdot10^{6} $ atoms per filling and the Avogadro constant $ \mathrm{N}_\mathrm{A} \approx 6.02\cdot10^{23}\,\mathrm{mol}^{-1} $ yields a femtomol level after 1000 fillings.
At this level, the macroscopic properties of Xenon are not altered by the remaining Krypton.


% Kr sample waveform
\begin{figure}
    \centering
    \includegraphics[width=0.95\textwidth]{Figures/th.jpeg}  % {Figures/waveform_kr.png} TODO make this plot and save it to figs-dir
    \caption[Waveform Krypton]{
        sample waveform of krm2 to krm1 and krm1 to kr. s1, s2... etc.
    }
    \label{fig:waveform_kr}
\end{figure}


% Nest scint response
\begin{figure}
    \centering
    \includegraphics[width=0.95\textwidth]{Figures/nest_scint_response.png}  % {Figures/th.jpeg}
    \caption[Nest scint response]{
        Nest scint response.\cite{Szydagis13}
    } % TODO write subbing
    \label{fig:scint_response_nest}
\end{figure}


With the $ 2.5\cdot10^{6} $ initial radioactive nuclei of Krypton, after 7 half-lives less than 2500 nuclei that have not decayed are left.
This corresponds to a sub-percent level after less than \SI{13}{\hour}.
After a day we can thus easily take data without a noticeable amount of Krypton decays occurring.

% TODO the following has been written before the text above. it might need to be worked on again in order to not repeat to much.
% also the desired rate is calculated differently. needs to be changed.
\paragraph{A Krypton run} is a \gls{daq} time for which we fill the \gls{tpc} with the metastable Krypton Isotope \isotope[83m][36]{Kr} via the Rubidium source in the gas system as described above.
The amount of Krypton filled into the \gls{tpc} and thus the activity, cannot be controlled precisely.
This is however not necessary as the activity decreases quickly due to the short decay time.
The signals of two or more interactions at the same time - meaning an interaction happened before or just at the same time as the previous \st~is registered - cannot unambiguously be assigned to its correct physical counterpart.
As we have not investigated how efficient these events can be removed, we want to keep their occurrence small compared to the occurrences of single interactions.
We aim for a maximum of about $ 1\,\% $ of the events being in coincidence.


%
% \begin{equation}
%     P_{\lambda}\left(k\right) =  \frac{\lambda^k}{\exp^{-\lambda}}
% \end{equation}
%

One minute after starting the filling process of Krypton for a Krypton run it reaches the inner part of the \gls{tpc}.
We allow the Krypton to disperse in the liquid Xenon for five minutes to ensure that it is distributed homogeneously inside the \gls{tpc}.
The homogeneity of the distribution is shown in subsection~\ref{ssec:tagging}.



\FloatBarrier
