
\FloatBarrier
\subsection{source}
\label{ssec:source}
\FloatBarrier


Don't forget to write these things(!!!):
\begin{enumerate}
    \item{Krypton nomenclature: Krypton run, Krypton event, Krypton ...}
    \item{$S1_a$, $S1_b$, etc.}
    \item{\gls{s1a}}
    \item{\gls{s1b}}
    \item{\gls{s2a}}
    \item{\gls{s2b}}

\end{enumerate}



describe how we can fill the tpc with krypton here.. whole cycle: rubidium, gas system, etc..

AT THE END:

\paragraph{Krypton runs} are what we henceforth call runs we filled \gls{tpc} with the meta stable Krypton Isotope for.
We aim to record a maximum number of these Krypon decays in the \SI{3}{\min} recordings.
The signals of two or more interactions at the same time - meaning an interaction happend before or just at the same time as the previous \st~ is registered - cannot unambiguously assigned to its correct physical counterpart.
Therefore, we can approximate an optimal rate of interactions and accordingly activity of our Krypton source $ A_\mathrm{Kr} $ inside the \gls{tpc}.
To do that we demand the probability of exactly one decay within a given time window to be maximal.
With the Krypton being distributed homogeneously within the \gls{tpc} the average \st~ drift time is half the maximum drift time of about $ \SI{20}{\micro\s}$.
Thus the optimal activity of Krypton can be calculated using Poissonian statistics with $\lambda = \SI{20}{\micro\s} \cdot A_\mathrm{Kr}$ and $ k = 1 $:

\begin{equation}
    P_{\lambda}\left(k\right) =  \frac{\lambda^k}{\exp^{-\lambda}}
\end{equation}

The highest probability to have exactly one decay within the average drift time of \SI{20}{\micro\s} is exactly \SI{50}{\kilo\becquerel}.
Although \SI{40}{\micro\s} is a good estimate, our maximum drift time depends on the drift field and can thus vary from run to run, depending on the field we choose.
It is also not easy to adjust how much Krypton exactly is let into the \gls{tpc} and thus the activity inside.
This is not necessary, because of the short decay time the activity will decrease quickly.
At the same time the probability around the maximum is rather flat so the amount of events that can be unambiguously reconstructed does not change drastically.
On the on hand, lesser activity will lead to less decays happening at the same time, but at the same time to a lot of time in which no decay will occure.
On the other hand, a greater activity increases overlapping events.
This calculation assumes having overlapping events is as unappreciated as having no data in a certain time slot.
Overlapping events can lead to mistakes in the reconstruction and are thus rather to be avoided than the other case.
The calculation also assumes only Krypton interactions to occur, whereas we know we also have other signal happening simultaneously.
For these reasons we choose a activities smaller than the theoretical optimum.


One minute after starting the filling process of Krypton for a Krypton run it reaches the inner part of the \gls{tpc}.
We allow the Krypton to disperse in the liquid Xenon for five minutes to ensure that is distrubuted homogeneously inside the \gls{tpc}.
That this is enough time to ensure a homogeneous distribution will be shown in subsection~\ref{ssec:tagging}.


















\FloatBarrier
