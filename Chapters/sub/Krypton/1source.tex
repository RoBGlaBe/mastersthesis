
\FloatBarrier
\subsection{source}
\label{ssec:source}
\FloatBarrier


Rubidium \isotope[83][37]{Rb} decays via two metastable excited states to the stable \isotope[83][36]{Kr} Krypton isotope.
It first transitions via electron capture to \isotope[83m2]{Kr}.
This state has a half-life of \SI{1.83}{\hour} and decays via an energy emission of \SI{32.1}{\kilo\electronvolt}, see fig.~\ref{fig:scheme_kr}.
Under the emission of an electron from internal conversion and subsequent emission of auger electrons and small amounts of X-rays its core is left as another exited state: \isotope[83m1]{Kr}.
The de-excitation to the stable \isotope[83]{Kr} is mostly via internal conversion and in small amounts via gamma-rays.
The half-life of the latter is \SI{154}{\nano\s} and the decay energy is \SI{9.4}{\kilo\electronvolt}.


% Krypton Decay Scheme
\begin{figure}
    \centering
    \includegraphics[width=0.95\textwidth]{Figures/th.jpeg}  % {Figures/scheme_kr_decay.png}
    \caption[Decay Scheme Krypton]{Kr-decay scheme\cite{kr_scheme}.}
    \label{fig:scheme_kr}
\end{figure}

% want kr inside tpc
The solid Rubidium source is placed in a reservoir inside the gas system.
The flow can be directed through the reservoir or past it without contact.
When the flow follows the line that goes through the reservoir the gaseous \isotope[83m2]{Kr}, stemming from the Rubidium decay, is flushed along with the Xenon into the \gls{tpc}.
There, it disperses and is spread homogeneously within approximately a minute.
Once it decays, the second decay takes place virtually immediately, both resulting in \gls{lxe} scintillation, respectively, see sec.~\ref{sec:Xebra}.
Fig.~\ref{fig:waveform_kr} shows a sample waveform of this process.
The different channels of both decays - even though the energy is the same - result in slightly different light and electron responses in the scintillation process. (TODO: scint response too slangy?)
This is shown in fig.~\ref{fig:scint_response_nest}.
We are not able to resolve these subtle differences, as other statistical effects in the detection process is the resolution.
It will, however, contribute to the broadening of the energy resolution. (TODO: nicht richtig belegbar...)
As the final decay product is \isotope[83]{Kr}, a stable Noble gas, it does neither react with other material, nor contribute background radiation and can thus be left in the Xenon.
A very conservatively estimated \SI{50000}{} atoms per filling and the Avogadro constant $ \mathrm{N}_\mathrm{A} \approx 6.02\cdot10^{23}\,\mathrm{mol}^{-1} $ yields a sub femto mol level after 1000 fillings.
At this level the macroscopic properties of Xenon are not altered by the remaining Krypton.


% Kr sample waveform
\begin{figure}
    \centering
    \includegraphics[width=0.95\textwidth]{Figures/th.jpeg}  % {Figures/waveform_kr.png} TODO make this plot and save it to figs-dir
    \caption[Waveform Krypton]{sample waveform of krm2 to krm1 and krm1 to kr. s1, s2... etc.}
    \label{fig:waveform_kr}
\end{figure}


% Nest scint response
\begin{figure}
    \centering
    \includegraphics[width=0.95\textwidth]{Figures/th.jpeg}  % {Figures/nest_scint_response.png}
    \caption[Nest scint response]{Nest scint response\cite{}.} % TODO cite nest paper and write subbing
    \label{fig:scint_response_nest}
\end{figure}


With the \SI{50000}{} initial radioactive nuclei of Krypton, after 7 half-lives less than 400 nuclei that have not decayed are left.
This corresponds to a sub-percent level after less than \SI{13}{\hour}. %  \footnote{TODO: Sounds weird to mention half-life in the same sentence as half-lifes and hours in the same sentences as per-cent-level, but I don't want to sound stupid by saying sub-percent corresponding to 400 nuclei. this is the true value after 7 half-lives, but doesnt correspond to the percent level, which seems to be what I'm implying then. on the other hand I want to make clear, that we always have a sub-percent level after 7 half-lives, that doesnt depend on the half-life. ideas? or maybe just leave like that?}
After a day we can thus easily take data without a noticeable amount of Krypton decays occurring.

% TODO the following has been written before the text above. it might need to be worked on again in order to not repeat to much.
% also the desired rate is calculated differently. needs to be changed.
\paragraph{A Krypton run} is a run for which we fill the \gls{tpc} with the metastable Krypton Isotope \isotope[83m2][36]{Kr} via the Rubidium source in the gassystem as described above.
We aim to record a maximum number of these Krypton decays in the \SI{3}{\min} recordings.
The signals of two or more interactions at the same time - meaning an interaction happened before or just at the same time as the previous \st~ is registered - cannot unambiguously be assigned to its correct physical counterpart.
Therefore, we can approximate an optimal rate of interactions and accordingly activity of our Krypton source $ A_\mathrm{Kr} $ inside the \gls{tpc}.
To do that we demand the probability of exactly one decay within a given time window to be maximal.
With the Krypton being distributed homogeneously within the \gls{tpc} the average \st~ drift time is half the maximum drift time of about $ \SI{20}{\micro\s}$.
Thus the optimal activity of Krypton can be calculated using Poissonian statistics with $\lambda = \SI{20}{\micro\s} \cdot A_\mathrm{Kr}$ and $ k = 1 $:

\begin{equation}
    P_{\lambda}\left(k\right) =  \frac{\lambda^k}{\exp^{-\lambda}}
\end{equation}

The highest probability to have exactly one decay within the average drift time of \SI{20}{\micro\s} is exactly \SI{50}{\kilo\becquerel}.
Although \SI{40}{\micro\s} is a good estimate, our maximum drift time depends on the drift field and can thus vary from run to run, depending on the field we choose.
It is also not easy to adjust how much Krypton exactly is let into the \gls{tpc} and thus the activity inside.
This is not necessary, because of the short decay time the activity will decrease quickly.
At the same time the probability around the maximum is rather flat so the amount of events that can be unambiguously reconstructed does not change drastically.
On the on hand, lesser activity will lead to less decays happening at the same time, but at the same time to a lot of time in which no decay will occur.
On the other hand, a greater activity increases overlapping events.
This calculation assumes having overlapping events is as unappreciated as having no data in a certain time slot.
Overlapping events can lead to mistakes in the reconstruction and are thus rather to be avoided than the other case.
The calculation also assumes only Krypton interactions to occur, whereas we know we also have other signal happening simultaneously.
For these reasons we choose a activities smaller than the theoretical optimum.


One minute after starting the filling process of Krypton for a Krypton run it reaches the inner part of the \gls{tpc}.
We allow the Krypton to disperse in the liquid Xenon for five minutes to ensure that is distributed homogeneously inside the \gls{tpc}.
That this is enough time to ensure a homogeneous distribution will be shown in subsection~\ref{ssec:tagging}.



\FloatBarrier
