
\FloatBarrier
\subsection{width}
\label{ssec:width}
\FloatBarrier


Once extracted into the gas phase the electron has a certain probability to emit a photon per path length traveled until the anode is reached. % TODO quelle angeben
With a given drift velocity the time of possible emittance is hence fixed by the total path length.
This is an upper limit for the length or \SI{100}{\%} width of an \gls{se} peak and evidently also other width quantiles.
As the width is so sensitive on the size of the gas gap the width is a good indicator for deformations of the grid in case a position reconstruction of of \gls{se}s is possible.

S2s usually have $ z $ dependent widths due to longitudinal diffusion.
The arrival time at the liquid interface of the electrons in one S2 electron clound spreads dependent on the drift time.
For \gls{se} this is not the case since these are single quanta.
\gls{se}s, however, tell us the intrinsic width of one electron which can be used to calculate the exact time between the first and last electron in the electron cloud.


Ohne eine Analyse sollte ich das vielleicht eher in ein gemeinsames kapitel mit der rate stecken und sowas wie outlook drauß machen?









