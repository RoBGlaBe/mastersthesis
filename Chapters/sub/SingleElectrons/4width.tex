
\FloatBarrier
\subsection{width}
\label{ssec:width}
\FloatBarrier


Once extracted into the gas phase an electron has a certain probability to emit a photon per path length traveled until it reaches the anode. % TODO quelle angeben
With a given drift velocity, the time of possible emittance is hence fixed by the total path length.
This time interval is an upper limit for the length, or \SI{100}{\%} width of a \gls{se} peak, and evidently also other width quantiles.
As the width is very sensitive to the size of the gas gap, the width is a good indicator of deformations of the grid.
As discussed, a spacial mapping is not feasible.

% S2s usually have $ z $ dependent widths due to longitudinal diffusion.
% The arrival time at the liquid interface of the electrons in one S2 electron clound spreads dependent on the drift time.
Due to longitudinal diffusion, S2 widths depend, among others, on the drift time, or interaction depht $ z $.
The spread of the electron cloud increases accordingly with $ z $ when arriving at the liquid interface.
For \gls{se} this is not the case since these are single quanta.
\glspl{se}, however, tell us the intrinsic width of one electron which can be used to calculate the exact time between the first and last electron in the electron cloud.


Ohne eine Analyse sollte ich das vielleicht eher in ein gemeinsames kapitel mit der rate stecken und sowas wie outlook drauß machen?









