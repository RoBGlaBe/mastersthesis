
\FloatBarrier
\subsection{Single Electron Production}
\label{ssec:se-source}
\FloatBarrier


A \gls{se} is one electron that is extracted from \gls{lxe} into the gas phase where it produces scintillation light along its path to the anode.
Once extracted into the gas phase an electron has a certain probability to emit a photon per path length traveled until it reaches the anode. % TODO quelle angeben
With a given drift velocity, the time of possible emittance is hence fixed by the total path length, does however fluctuate due to longitudinal diffusion.
This time interval is an upper limit for the length, or \SI{100}{\%} width of a \gls{se} peak, and evidently also other width quantiles.
The amount of light produced in this process is dominated by the field strength of the amplification field $ E_\mathrm{amp} $, the size of the gas gap, and the gas pressure.

The supposedly main source of \glspl{se} is photoionization of impurities in the \gls{lxe} like water and $ 0_2 $.
These strongly electronegative components can capture an electron from, e.g., the electron-ion production in the Xenon scintillation process.
Light from an S1 or S2 can break the weak bond of the electron and the impurity and thus free the captured electron.
We refer to this process loosely as photoionization knowingly disregarding that no ion is formed in this process.
If it is not captured again, it reaches the liquid surface and undergoes proportional scintillation if it is extracted into the gas phase.
The longer an electron drifts, the more likely it is recaptured.
Therefore, the \gls{se} rate drops with the time since last S2, which relates to the depth the electron originates from, as shown in fig.~\ref{fig:se-rate-xe100}.  % TODO add one of the plots of 1T, Xe100 or LUX from my SE talk
An S2 produces more light than the related S1 and thus causes more photoionizations.
The \gls{se} rate increases with larger S2 signals, i.e. higher energetic interactions and also with larger amounts of impurities\cite{?}.  % TODO cite paper
Next to the photoionization of impurities, there are other, secondary sources of \glspl{se} like late extraction of electrons from the liquid surface.


% SE rate 1T, 100 or LUX
\begin{figure}
    \centering
    \includegraphics[width=0.95\textwidth]{Figures/se_rate_xe100.png}  % {Figures/th.jpeg}
    \caption[Single Electron Rate in 1T/100/LUX]{
        Single Electron rate in 1T/100/LUX. Currently Xenon100. Other? 2? All? Drawback with 1T: not a paper, but internal note but I think blessed...\cite{Aprile2014}.
    }
    \label{fig:se-rate-xe100}
\end{figure}

