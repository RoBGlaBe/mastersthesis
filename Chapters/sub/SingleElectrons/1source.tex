
\FloatBarrier
\subsection{source}
\label{ssec:source}
\FloatBarrier


An \gls{se} is one electron that is extracted from \gls{lxe} into the gas phase and amplified.
Light from S1 and S2 can photoionize impurities in the Xenon that captured electrons previously.
If they are not recaptured by an impurity they reach the liquid surface and undergo proportional scintillation in the case it is extracted into the gas phase.
The longer an \gls{se} drifts, the more likely it is that it is recaptured.
Therefore, the \gls{se} rate drops with depth or rather with time after the last S2, as shown in fig.~\ref{fig:se-rate}.  % TODO add one of the plots of 1T, Xe100 or LUX from my SE talk
An S2 produces more light than the related S1 and has thus more potential to ionization.
The \gls{se} increases with larger S2 signals, i.e. higher energetic interactions and also with larger amounts of impurities\cite{}.  % TODO cite paper
Next to the ionization of impurities there are other, secondary sources of single electron like late extraction of electrons from the liquid surface.


% SE rate 1T, 100 or LUX
\begin{figure}
    \centering
    \includegraphics[width=0.95\textwidth]{Figures/th.jpeg}  % {Figures/se_rate.png}
    \caption[Single Electron Rate in 1T/100/LUX]{
        Single Electron rate in ...\cite{}.
    }
    \label{fig:se-rate}
\end{figure}


% TODO write which kind of run we use (krypton run 163)


