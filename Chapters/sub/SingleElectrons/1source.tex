
\FloatBarrier
\subsection{Single Electron Production}
\label{ssec:se-source}
\FloatBarrier


A \gls{se} is one electron that is extracted from \gls{lxe} into the gas phase and is then amplified.
Light from S1s and S2s can photoionize impurities in \gls{lxe} that previously captured electrons.
If they are not captured by an impurity, they reach the liquid surface and undergo proportional scintillation in the case it is extracted into the gas phase.
The longer a \gls{se} drifts, the more likely it is recaptured.
Therefore, the \gls{se} rate drops with depth $ z $ or rather with time since last S2, as shown in fig.~\ref{fig:se-rate-xe100}.  % TODO add one of the plots of 1T, Xe100 or LUX from my SE talk
An S2 produces more light than the related S1 and has thus more potential to ionize.
The \gls{se} increases with larger S2 signals, i.e. higher energetic interactions and also with larger amounts of impurities\cite{?}.  % TODO cite paper
Next to the ionization of impurities, there are other, secondary sources of \gls{se} like late extraction of electrons from the liquid surface.


% SE rate 1T, 100 or LUX
\begin{figure}
    \centering
    \includegraphics[width=0.95\textwidth]{Figures/se_rate_xe100.png}  % {Figures/th.jpeg}
    \caption[Single Electron Rate in 1T/100/LUX]{
        Single Electron rate in 1T/100/LUX. Currently Xenon100. Other? 2? All? Drawback with 1T: not a paper, but internal note but I think blessed...\cite{Aprile2014}.
    }
    \label{fig:se-rate-xe100}
\end{figure}


% TODO write which kind of run we use (krypton run 163)


