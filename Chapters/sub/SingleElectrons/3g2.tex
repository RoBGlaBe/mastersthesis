
\FloatBarrier
\subsection{Spatial Gain Dependence}
\label{ssec:g2}
\FloatBarrier

We fit all the resulting populations in an area histogram with % the model $ f_\mathrm{SE}\left( \ar \right) $

% se len cut
\begin{figure}
    \centering
    \includegraphics[width=\figw\textwidth]{Figures/se_cut_length.png}  % {Figures/th.jpeg}
    \caption[Length Histogramm and Cut for SEs]{
        Length Histogram of the previously defined reference regimes.
        The different reference regimes populate different length zones.
        This is used to introduce a cut to separate the background from the other two populations.
        The tail regime corresponds to coinciding few-electron signals and is thus later used to constrain the fit of the \gls{se} population.
        We therefore purposefully do not cut the tail but only define a minimal-length cut for the background indicated by the vertial blue line at \SI{350}{\nano\second}.
    }
    \label{fig:len-cut-se}
\end{figure}


% se n-hits cut
\begin{figure}
    \centering
    \includegraphics[width=\figw\textwidth]{Figures/se_cut_n_hits.png}  % {Figures/th.jpeg}
    \caption[N-Hits Histogramm and Cut for SEs]{
        N-Hits Histogram of the previously defined reference regimes.
        The different reference regimes populate different length zones.
        This is used to introduce a cut to separate the background from the other two populations.
        The tail regime corresponds to coinciding few-electron signals and is thus later used to constrain the fit of the \gls{se} population.
        We therefore purposefully do not cut the tail but only define a minimal-n-hits cut for the background indicated by the vertial blue line at $ 5\,\mathrm{hits} $.
    }
    \label{fig:hit-cut-se}
\end{figure}


% se risetime cut
\begin{figure}
    \centering
    \includegraphics[width=\figw\textwidth]{Figures/se_cut_risetime.png}  % {Figures/th.jpeg}
    \caption[Risetime Histogramm and Cut for SEs]{
        Risetime Histogram of the previously defined reference regimes.
        The different reference regimes populate different length zones.
        This is used to introduce a cut to separate the background from the other two populations.
        The tail regime corresponds to coinciding few-electron signals and is thus later used to constrain the fit of the \gls{se} population.
        We therefore purposefully do not cut the tail but only define a minimal-risetime cut for the background indicated by the vertial blue line at \SI{35}{\nano\second}.
    }
    \label{fig:riset-cut-se}
\end{figure}


% f_\mathrm{SE}\left( \ar \right) =
\begin{align}
    &\sum_{n} G \left(  \ar; n \cdot g_\mathrm{SE}, \sqrt{n} \cdot \sigma_\mathrm{SE}, A_n \right) +
    G_\mathrm{bg} \left( \ar;  \mu_\mathrm{bg}, \sigma_\mathrm{bg}, A_\mathrm{bg} \right), \\
    &\mathrm{where~} G \left( x;  \mu, \sigma, A \right) = A \cdot \exp{ \left\{ \frac{-\left( x - \mu \right)^2}{2\cdot\sigma^2} \right\} }
    \label{eq:se-fit-model}
\end{align}


is a Gaussian function, with $ n $ being the number of simultaneously detected electrons (here \numrange{1}{4}).
The parameter estimates of $ g_\mathrm{SE} $ by the fits in fig.~\ref{fig:se-area-fits} are hardly influenced by the different background cuts.
As the parameter is stable, we intend to reduce the background as much as possible for other analyses.
Our background cut hence is the n-hits cut, fig.~\ref{fig:hit-cut-se}.
The resulting fit parameter value is $ g_\mathrm{SE} = (22.11 \pm 0.05)\,\mathrm{PE} $.
The parameter is stable within $ 0.3\,\%_\mathrm{rel} $ in variation of the last background cut.


% area hist fit for g_se
\begin{figure}
    \centering
    \includegraphics[width=\figw\textwidth]{Figures/se-area-fits.png}  % {Figures/th.jpeg}  % TODO redo plot
    \caption[Area Histogram Fit of SEs]{
        Area histogram of the final \gls{se} population.
        The histogram includes the best fit of eq.~\ref{eq:se-fit-model} to the data between the two red lines indicating the fit window.
        A solid, orange line shows the best fit.
        The contributing Gaussian functions are shown separately as non-solid lines.
    }
    \label{fig:se-area-fits}
\end{figure}

With the same approach, we also investigate runs at different $ E_\mathrm{amp} $.
We compare the behavior to the one of \oneton~in fig.~\ref{fig:seg-vs-field}.
We note a systematically inferior light response of \gls{xebra}.  % TODO rephrase. we dont have a curve for xebra... just points
This can be explained by the different \gls{pmt} models of the top \gls{pmt} array, where the ones in \oneton~have the higher \glspl{qe} and \glspl{ce}.
The leveling of the liquid level can be conducted more precisely in a larger \gls{tpc} as well as better \gls{ptfe} reflectors are used in \oneton~ which increases the \gls{lce}.
% We measure $ g_\mathrm{SE, bot} = \,\mathrm{PE} $ at $ E_\mathrm{amp} \approx \SI{10.7}{\kilo\volt\per\centi\meter} $.
% We expect the signal seen in the bottom \gls{pmt} to be more comparable as we are using the better performing \gls{pmt} model and have a $ 100\,\% $ \gls{pmt} coverage at the bottom.


% g_se vs amp field
\begin{figure}
    \centering
    \includegraphics[width=\figw\textwidth]{Figures/se_gain_vs_field.png}  % {Figures/th.jpeg}
    \caption[\oneton~comparison of Amplification Gain vs. Fieldstrength]{
        $ g_\mathrm{SE} $ vs. $ E_\mathrm{amp} $
        % mention cite!!
    }
    \label{fig:seg-vs-field}
\end{figure}


Another issue we are facing lies in an earlier processing step of \emph{peaks}, the data kind used for this analysis.
In the waveform example in fig.~\ref{fig:waveforms-se}, we, on the one hand, note a frequent occurrence of samples with negative values and on the other hand a waveform that has seemingly arbitrarily split into two peaks.
The first issue can likely be traced back to the baseline reconstruction.
The second issue can likely be solved by refining the rules for peak splitting and peak merging.
Optimization of both these issues are possible, they are, however, out of the scope of this work.
While the first issue is responsible for an underestimate of nearly all peaks, the second can split up one \gls{se} with the full area into two which then share the total area of one physical \gls{se}.
Both effects reduce the value of $ g_\mathrm{SE} $ and the parameter is consequently underestimated.


% SE waveform problems
\begin{figure}
    \centering
    \includegraphics[width=\figw\textwidth]{Figures/th.jpeg}  % {Figures/se_waveforms.png}
    \caption[Single Electrons Waveforms]{
        Single Electron Waveforms
    }
    \label{fig:waveforms-se}
\end{figure}




It would be desirable to be able to resolve $ g_\mathrm{SE} $ spatially as a function of $ x $ and $ y $.
We would then be able to confirm a uniform amplification and to account for the irregularities.
As the spacial resolution is low for small signals~\cite{ABism} - the neural net is trained for $ > 100\,\mathrm{PE} $ - the position reconstruction is too imprecise in a small scale \gls{tpc} as \gls{xebra} to get a meaningful result.\footnote{auf alex thesis verweisen ohne weitere daten zu zeigen? haben wir nicht wirklich gecheckt, sondern uns auf darryls aussage verlassen}.



