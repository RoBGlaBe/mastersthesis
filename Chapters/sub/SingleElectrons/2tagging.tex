
\FloatBarrier
\subsection{Single Electron Peak Selection}
\label{ssec:tagging}
\FloatBarrier


We discuss our peak selection cuts and \gls{se} selection here based on the example run 163.
We aim for a high sample purity with some compromise towards a large sample size.
The latter can be further improved by analyzing more data, e.g., longer runs.
Further reference to our \gls{se} population will be based on the selection discussed here.

\paragraph{Time after S2 Selection:} The longest drift of an electron occurs when it emerges just above the cathode.
Most \glspl{se} are thus expected within one drift length after large S2s, as photoionization by S2 light yields most \glspl{se}.
We therefore select all peaks in the window of one drift length after every large S2 peak in the run.
A large S2 in such a window closes the window early and the newly found S2 opens another window as usual.

\paragraph{Two Channels Cut:} By introducing a cut based on the field \emph{n channels}, we select peaks with a specific number of \glspl{pmt} contributing to it.
We demand at least two contributing \glspl{pmt} and thereby suppress dark counts as their coincidence rate is low.
Real \glspl{se} are large enough to be seen by at least two \glspl{pmt} most of the time, as the scintillation light is emitted isotropically.
They are therefore not affected by the cut.

\paragraph{\gls{aft} cut:} Although the meaning of \glspl{se} and S2s differ, \glspl{se} are produced via the same scintillation process as S2s and thus share some essential properties, like the specific \gls{aft}.
As S2 photons are produced in a narrow z-range close to the top \glspl{pmt}, they have a certain probability to be detected by the top rather than the bottom \gls{pmt}.
The result is a S2-specific \gls{aft}, $ \mathit{AFT}_\mathrm{S2} $.
We can see in fig.~\ref{fig:se-aft} that in the \gls{xebra} dual-phase \gls{tpc} $ \mathit{AFT}_\mathrm{S2} \approx 0.35 $.
However, the smaller the signal, the more the value is dominated by statistical fluctuations.
\glspl{se} in that sense behave like small S2s and have therefore a wide spread of \glspl{aft}.
Cutting on the contours given by $ a + \nicefrac{b}{\sqrt{\mathit{area}}} $ does not improve our \gls{se} population significantly, but would rather help with larger S2s.
We get better results with a flat \gls{aft} cut $ \mathit{AFT}_\mathrm{min} = 0.05 $ and $ \mathit{AFT}_\mathrm{max} = 0.95 $ cutting bigger population clearly not being \glspl{se}.

% SE AFT
\begin{figure}
    \centering
    \includegraphics[width=\figw\textwidth]{Figures/se_aft.png}  % {Figures/th.jpeg}
    \caption[AFT vs Area Single Electrons]{
        2D-histogram of peaks binned linearly in \gls{aft} and logarithmically in area of a preselected \gls{se} population to introduce a cut based on \gls{aft}.
        The S2-specific \gls{aft} of $ \approx 0.35 $ for this \gls{tpc} unfolds for larger areas, while in the lower area regime of \gls{se} statistical fluctuations dominate the value, as indicated by the black, dotted line.
        Introducing a cut at the black line does not improve the population significantly.
        An \gls{aft} of 1 corresponds to a peak only seen by the \glspl{pmt} in the top array, whereas a \gls{aft} of 0 means that only the bottom \gls{pmt} detected the peak.
        We remove the larger population at 0 and the peaks with unphysical negative values by introducing a cut at the lower blue line at $ \mathit{AFT}_\mathrm{min} = 0.05 $.
        The upper blue line at $ \mathit{AFT}_\mathrm{max} = 0.95 $ introduces the upper \gls{aft} cut, removing the population at 1 and the unphysical values $ > 1 $.
    }
    \label{fig:se-aft}
\end{figure}

% TODO weiter oben reinschreiben, welche areas wir erwarten & welches feld wir nutzen
The area histogram up to \SI{100}{\mathit{PE}} with all the previous cuts applied, fig.~\ref{fig:se-area-hist}, shows one broader population at just above \SI{20}{\mathit{PE}} with a long tail.
We identify this as our \gls{se} population where we find simultaneous occurrences of multiple electrons in the tail.
Another slightly overlapping population at areas smaller than \SI{10}{\mathit{PE}} is considered background, originating, e.g., from coincident dark counts.
Before fitting the population to extract the mean as the \emph{amplification gain} $ g_\mathrm{SE} $, we want to make sure that the background population does not influence the result of the fit.
We can check this by reducing the background using more stringent cuts with higher \gls{se} purity but lower tagging efficiency and compare the fit results.
A stable value hints towards a low impact of the population on the fit.

% SE area hist
\begin{figure}
    \centering
    \includegraphics[width=\figw\textwidth]{Figures/se_area_hist.png}  % {Figures/th.jpeg}
    \caption[Histogram Area Single Electrons and Background]{
        Area histogram of our single electron population where the peak at $ 22\,\mathrm{PE} $ correspond to the alleged \glspl{se}.
        The tail is formed by coinciding few-electron peaks and the narrow peak on the left side at $ \approx 2\,\mathrm{PE} $ is considered background.
        Based on the area we define reference regimes indicated by the background colors.
        % That the background population peaks at $ \approx 2\,\mathrm{PE} $ suggests that it consists of 2 coinciding photons or dark counts.
        % In this figure peaks with only one contributing channel is excluded as are \gls{aft} values of 1 and 0.
        % This means that from the coinciding photons/dark counts from the background population one is seen by one top \gls{pmt} and one by the bottom \gls{pmt}.
    }
    \label{fig:se-area-hist}
\end{figure}


We choose the reference regimes \emph{background}, \emph{\glspl{se}}, and \emph{tail} based on areas.
The representative regions are indicated in fig.~\ref{fig:se-area-hist} by the different background colors spanning from \numrange{0}{7} (background, green), \numrange{15}{30} (\gls{se}, red) and \numrange{40}{100}$ \,\mathit{PE} $ (tail, orange).
Furthermore, we compare other dimensions of the reference regimes to each other to find one where the backgrounds is well separable from the other two.
We choose peak length, risetime, and number of hits which show different behavior in the different reference regimes and thus make for possible cuts.
The respective histograms and cuts are shown in fig.~\ref{fig:len-cut-se},~\ref{fig:hit-cut-se} and \ref{fig:riset-cut-se}.


%  TODO add table as proposed in Fabians corrections




