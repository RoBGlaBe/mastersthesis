
\FloatBarrier
\subsection{tagging}
\label{ssec:tagging}
\FloatBarrier


We discuss our data selection cuts and \gls{se} selection here.
Further reference to our \gls{se} population will be based on the selection discussed here.

\paragraph{After S2 Selection:} Most \gls{se} signals are expected within one drift length after a large S2 peak.
We therefore first select larger S2 peaks and further select every peak within one drift length after such an S2.
In case another large S2 peak occurs before the window of the previous S2 closes we open a new window at that point without selecting that S2 as a potential \gls{se}.

% \paragraph{Two Channels Cut:} We also demand that at least two \glspl{pmt} contribute to the peak.
% A peak from just one \gls{pmt} is very likely to stem from a dark count or noise.

\paragraph{Two Channels Cut:} By introducing a cut based on the field \emph{n channels}, we select peaks with a specific number of \glspl{pmt} contributing to it.
We demand at least two contributing \glspl{pmt} and thereby get rid of a lot of dark counts as they occur randomly and thus often not in coincidence.
Real \glspl{se} are large enough to be seen by at least two \glspl{pmt} most of the time and are therefore not affected by the cut.

\paragraph{\gls{aft} cut:} As S2 photons are produced in a narrow z-range close to the top \glspl{pmt}, they have a certain probability to be detected by the top rather than the bottom \gls{pmt}.
The result is a S2-specific \gls{aft}, $ \mathit{AFT}_\mathrm{S2} $.
We can see in fig.~\ref{fig:se-aft} that here $ \mathit{AFT}_\mathrm{S2} \approx 0.35 $.  % TODO: plot machen mit a+b/sqrt(area) und flat cuts: aft vs area
However, the smaller the signal, the more the value is dominated by statistical fluctuations.
As small S2s, \gls{se}, therefore, have a wide spread of \glspl{aft}.
Cutting on the contours given by $ a + \nicefrac{b}{\sqrt{\mathit{area}}} $ does not improve our \gls{se} population significantly, but would rather help with larger S2s.
We get better results with a flat \gls{aft} cut $ \mathit{AFT}_\mathrm{min} = 0.05 $ and $ \mathit{AFT}_\mathrm{max} = 0.95 $ cutting bigger population clearly not being \glspl{se}.

% SE AFT
\begin{figure}
    \centering
    \includegraphics[width=\figw\textwidth]{Figures/th.jpeg}  % {Figures/se_aft.png}
    \caption[AFT vs Area Single Electrons]{
        Single Electron AFT vs Area.
    }
    \label{fig:se-aft}
\end{figure}

% TODO weiter oben reinschreiben, welche areas wir erwarten & welches feld wir nutzen
TODO AB HIER INS G2 KAPITEL?
The area histogram up to \SI{100}{\mathit{PE}} with all the previous cuts applied, fig.~\ref{fig:se-area-hist}, shows one broader population at just above \SI{20}{\mathit{PE}} with a long tail.
We identify this as our \gls{se} population where we find simultaneous occurrences of multiple electrons in the tail.
Another slightly overlapping population at areas smaller than \SI{10}{\mathit{PE}} is background.
Before fitting the population to extract the mean as the \emph{amplification gain} $ g_\mathrm{SE} $, we want to make sure that the background population does not influence the result of the fit.
We can check this by reducing the background and compare the fit results.
A stable value hints towards a low impact of the population on the fit.

% SE area hist
\begin{figure}
    \centering
    \includegraphics[width=\figw\textwidth]{Figures/th.jpeg}  % {Figures/se_area_hist.png}
    \caption[Histogram Area Single Electrons and Background]{
        Single Electron AFT vs Area.
    }
    \label{fig:se-area-hist}
\end{figure}


We choose the representative groups \emph{background}, \emph{\glspl{se}}, and \emph{tail} based on areas.
The representative regions are indicated in fig.~\ref{fig:se-area-hist} by the different background colors spanning from \numrange{0}{7} (background, green), \numrange{15}{30} (\gls{se}, red) and \numrange{40}{100}$ \,\mathit{PE} $ (tail, orange).
Furthermore, we compare other fields of the representatives to each other to find one where the backgrounds is well separable from the other two.
The peak length, risetime, and number of hits make for possible cuts.
The respective histograms and cuts are shown in fig.~\ref{fig:len-cut-se},~\ref{fig:len-cut-se} and \ref{fig:len-cut-se}.
We fit all the resulting populations in an area histogram with % the model $ f_\mathrm{SE}\left( \ar \right) $


% f_\mathrm{SE}\left( \ar \right) =
\begin{align}
    &\sum_{n} G \left(  \ar; n \cdot g_\mathrm{SE}, \sqrt{n} \cdot \sigma_\mathrm{SE}, A_n \right) +
    G_\mathrm{bg} \left( \ar;  \mu_\mathrm{bg}, \sigma_\mathrm{bg}, A_\mathrm{bg} \right), \\
    &\mathrm{where~} G \left( x;  \mu, \sigma, A \right) = A \cdot \exp{ \left\{ \frac{-\left( x - \mu \right)^2}{2\cdot\sigma^2} \right\} }
    \label{eq:se-fit-model}
\end{align}


is a Gaussian function, for n from \numrange{1}{4}.
The parameter estimates of $ g_\mathrm{SE} $ by the fits in fig.~\ref{fig:se-area-fits} are hardly influenced by the different background cuts.
As the parameter is stable, we intend to reduce the background as much as possible for other analyses.
Our background cut hence is XXX.  % TODO cut value einfügen
We use the different results to estimate the systematic uncertainties.
The resulting fit parameter value is $ g_\mathrm{SE} = X \pm X_\mathrm{stat} \pm X_\mathrm{sys} $.  % TODO add value of g_{se}


With the same approach, we also investigate runs at different $ E_\mathrm{amp} $.
We compare the behavior to the one of \oneton~in fig.~\ref{fig:seg-vs-field}.
Although the curves are parallel, we note a systematically worse light response of \gls{xebra}.
This can be explained by the different \gls{pmt} models of the top \gls{pmt} array, where the ones in \oneton have the higher \glspl{qe} and \glspl{ce}.
Also, large efforts have been invested in \oneton, that are not feasible for \gls{xebra}, to achieve high light yields\footnote{am besten irgendwie anders verpacken?}.
We can overcome these differences by comparing the signal seen in the bottom \glspl{pmt}, since we use the same \gls{pmt} model there as in \oneton~and effectively go against light loss due to the shorter \gls{tpc}.
With $ g_\mathrm{SE, bot} = XX $ at $ E_\mathrm{amp} = \left( XX \pm XX \right) \SI{}{\kilo\volt\per\centi\meter} $, we are in a comparable regime to \oneton~within XXXX.  % TODO fill in blanks


% g_se vs amp field
\begin{figure}
    \centering
    \includegraphics[width=\figw\textwidth]{Figures/th.jpeg}  % {Figures/se_gain_vs_field.png}
    \caption[\oneton comparison of Amplification Gain vs. Fieldstrength]{
        $ g_\mathrm{SE} $ vs. $ E_\mathrm{amp} $
    }
\end{figure}
    \label{fig:seg-vs-field}


Another issue we are facing lies in an earlier processing step of peaks.
In the waveform example in fig.~\ref{fig:waveforms-se}, we, on the one hand, note a frequent occurrence of samples with negative values and on the other hand a waveform that has seemingly arbitrarily split into two peaks.
While the first issue is responsible for an underestimate of nearly all peaks, the second can split up one \gls{se} with the full area into two which then share the total area of one physical \gls{se}.
Both effects reduce the value of $ g_\mathrm{SE} $ and the parameter is consequently underestimated.


% SE waveform problems
\begin{figure}
    \centering
    \includegraphics[width=\figw\textwidth]{Figures/th.jpeg}  % {Figures/se_waveforms.png}
    \caption[Single Electrons Waveforms]{
        Single Electron Waveforms
    }
\end{figure}


